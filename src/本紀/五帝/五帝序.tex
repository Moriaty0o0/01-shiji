|[%
裴駰曰凡是徐氏義稱徐姓名以別之餘者悉是駰%
注解并集衆家義。司馬貞索隱曰紀者記也本其%
%
事而記之故曰本紀又紀理也絲縷有紀而帝王書%
稱紀者言爲後代綱紀也。正義曰鄭玄注中候勅%
%
省圖云德合五帝坐星者稱帝又坤靈圖云德配天%
地在正不在私曰帝按太史公依世本大戴禮以黃%
%
帝顓頊帝嚳唐堯虞舜爲五帝譙周應劭宋均皆同%
而孔安國尚書序皇甫謐帝王世紀孫氏注世本並%
%
以伏犧神農黃帝爲三皇少昊顓頊高辛唐虞爲五%
帝裴松之史目云天子稱本紀諸侯曰世家本者繫%
%
其本系故曰本紀者理也統理衆事繫之年月名之%
曰紀第者次序之目一者舉數之由故曰五帝本紀%
%
第一。又曰禮云動則左史書之言則右史書之正%
義云左陽故記動右隂故記言言爲尚書事爲春秋%
%
按春秋時置左右%
史故云史記也%
]%
