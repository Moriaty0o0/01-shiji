虞舜者|[%
謚法曰仁聖盛明曰舜。索隱曰虞國名在河東%
太陽縣舜謚也皇甫謐云舜字都君也。正義曰%
%
括地志云故虞城在陝州河北縣東北五十里虞山之上酈%
元注水經云幹橋東北有虞城堯以女𡣕于虞之地也又宋%
%
州虞城大襄國所封之邑杜預云舜後諸侯也又越州餘姚%
縣顧野王云舜後攴庶所封之地舜姚姓故云餘姚縣西七%支
%
十里有漢上虞故縣會稽舊記云舜上虞人去虞三十里有%
姚丘即舜所生也周處風土記云舜東夷之人生姚丘括地%
%
志又云姚墟在濮州雷澤縣東十三里孝%
經援神契云舜生於姚墟按二所未詳也%
]%
名曰重華|[%
徐%
廣%
%
曰皇甫謐云舜以堯之二十一年甲子生三十一年甲午徴%
用七十九年壬午即眞百歳癸卯崩。正義曰尚書云重華%
%
叶於帝孔安國云華謂文德也言其光又重合於堯瞽叟姓%
嬀妻曰握登見大虹意感而生舜於姚墟故姓姚目重童子%
%
故曰重華字都君龍顏大%
口黑色身長六尺一寸%
]%
重華父曰瞽叟|[%
正義曰先后%
反孔安國云%
%
無目曰瞽舜父有目不能分別好惡故%
時人謂之瞽配字曰叟叟無目之稱也%
]%
瞽叟父曰橋牛|[%
正義曰橋%
又音嬌%
]%
橋牛父曰句望|[%
正義句古侯%
反望音亡%
]%
句望父曰敬%
康敬康父曰窮蟬窮蟬父曰帝顓頊顓頊父曰%
昌意以至舜七世矣自從窮蟬以至帝舜皆㣲%
爲庶人舜父瞽叟盲而舜母死|[%
索隱曰皇甫謐云%
舜母名握登生舜%
%
於姚墟因%
姓姚氏也%
]%
瞽叟更娶妻而生象象傲瞽叟愛後%
妻子常欲殺舜舜避逃及有小過則受罪順事%
父及後母與弟日以篤謹匪有懈舜兾州之人%
也|[%
正義曰蒲州河東縣本屬冀州宋永初山川記云蒲坂城%
中有舜廟城外有舜宅及二妃檀括地志云嬀州有嬀水%
%
源岀城中耆舊傳云即舜釐降二女於嬀汭之所外城中有%
舜井城城北有歷山山上有舜廟未詳按嬀州亦冀州城是%
%
也%
%
]%
舜耕歷山|[%
鄭玄曰在河東。正義曰括地志云蒲州河%
東縣雷首山一名中條山亦名歷山亦名首%
%
陽山亦名蒲山亦名襄山亦名甘棗山亦名猪山亦名狗頭%
山亦名薄山亦名吴山此山西起雷首山東至吳坂凡十二%
%
名隨州縣分之歷山南有舜井又云越州餘姚縣有歷山舜%
井濮州雷澤縣有歷山舜井二所又有姚墟云生舜處也及%
%
嬀州歷山舜井皆云%
舜所耕處未詳也%
]%
漁雷澤|[%
鄭玄曰雷夏兖州澤今屬濟%
隂。正義曰括地志云雷夏%
%
澤在濮州雷澤縣郭外西北山海經云%
雷澤有雷神龍身人類鼓其腹則雷也%頭
]%
陶河濵|[%
皇甫謐曰%
濟隂定陶%
%
西南陶丘亭是也。正義曰按於曹州濵河作瓦器也括地%
志云陶城在蒲州河東縣北三十里即舜所都也南去歷山%
%
不逺或陶所在則可何必定陶%
方得爲舜陶之陶也斯或一焉%
]%
作什器於壽丘|[%
皇甫謐%
曰在魯%
%
東門之北。索隱曰什器什數也盖人家常用之器非一故%
以十爲數猶今云什物也壽丘地名黄帝生處。正義曰壽%
%
音受顏師古云軍法伍人爲伍二伍爲什則共器物故爲生%
生之具爲什器亦猶從軍及作役者十人爲火共畜調度也%
]%
就時於負夏|[%
鄭玄曰負夏衛地。索隱曰就時猶逐時%
若言乗時射利也尚書大傳曰販於頓丘%
%
就時負夏孟子曰%
遷于負夏是也%
]%
舜父瞽叟頑母嚚弟象傲皆欲%
殺舜舜順適不失子道兄弟孝慈欲殺不可得%
即求甞在側舜年二十以孝聞三十而帝堯問%
可用者|[%
正義曰可用謂%
可爲天子也%
]%
四嶽咸薦虞舜曰可於是%
堯乃以二女妻舜以觀其內使九男與處以觀%
其外舜居嬀汭內行彌謹堯二女不敢以貴驕%
事舜親戚|[%
正義曰二女不敢以帝女驕慢舜之親戚親戚%
謂父瞽叟後母弟象妹顆手等也顆音古果反%
]%
甚有婦道堯九男皆益篤|[%
正義曰篤悙也非唯二女%惇
恭勤婦道九男事舜皆益%
%
悙厚謹%惇
敬也%
]%
舜耕歷山歷山之人皆讓畔|[%
正義曰韓子曆%
云農相侵略舜%
%
往耕朞年耕%
者讓畔也%
]%
漁雷澤雷澤上人皆讓居陶河濵河%
濵器皆不苦窳|[%
史記音隱曰音游甫反駰謂寙病也。%
正義曰苦讀如盬音古盬麤也寙音庾%
]%
一年而所居成聚|[%
正義曰聚在喻%
反聚謂村落也%
]%
二年成邑三年%
成都|[%
正義曰周禮郊野法云九夫爲井四井爲邑%
四邑爲丘四丘爲甸四甸爲縣四縣爲都也%
]%
堯乃%
賜舜絺衣|[%
正義曰絺勑遟反細葛%
布衣也鄒氏音竹几反%
]%
與琴爲築倉廪%
予牛羊瞽叟尚復欲殺之使舜上塗廪瞽叟從%
下縱火焚廪舜乃以兩笠自扞而下去得不死%
|[%
索隱曰言以笠自扞己身有似鳥張翅而輕下得不損傷皇%
甫謐云兩繖繖笠類列女傳云二女教舜鳥工上廪是也。%
%
正義曰通史云瞽叟使舜滌廪舜告堯二女女曰%
時其焚汝鵲汝衣裳鳥工往舜旣登廪得免去也%
]%
後瞽叟%
又使舜穿井舜穿井爲匿空旁岀|[%
劉熙曰舜以權%
謀自免亦大聖%
%
有神人之助也。索隱曰空音孔列女傳所謂龍工入井是%
也。正義曰言舜僭匿穿孔旁從他井而岀也通史云舜穿%潛
%
井又告二女二女曰去汝裳衣龍工往入井瞽叟與象下土%
實井舜從他井岀去也括地志云舜井在嬀州懷戎縣西外%
%
城中其西又有一井耆舊傳云並舜井也%
舜自中岀帝王紀云河東有舜井未詳也%
]%
舜旣入深瞽%
叟與象共下土實井|[%
索隱曰亦%
作塡井%
]%
舜從匿空岀去%
瞽叟象喜以舜爲已死象曰本謀者象象與其%
父母分|[%
正義扶%
問反%
]%
於是曰舜妻堯二女與琴象取之%
牛羊倉廪予父母象乃止舜宮居|[%
正義曰宮即室%
也爾雅云室謂%
%
之宮禮云命士已%
上父子異宮也%
]%
鼓其琴舜徃見之象鄂不懌曰%
我思舜正鬱陶舜曰然爾其庶矣|[%
索隱曰言汝猶%
當庶幾於友悌%
%
之情義也如孟子取尚書文又云惟茲臣%
庶女其于予治盖欲令象共我理臣庶也%
]%
舜復事瞽叟%
愛弟彌謹於是堯乃試舜五典百官皆治昔髙%
陽氏有才子八人|[%
名見%
左傳%
]%
世得其利謂之八愷|[%
賈%
逵%
%
曰愷和也。索隱曰左傳史克對季文子曰昔髙陽氏%
有才子八人倉舒隤皚檮戭大臨厖降庭堅仲容叔達%
]%
髙辛%
氏有才子八人|[%
名見%
左傳%
]%
世謂之八元|[%
賈逵曰元善也。%
索隱曰左傳髙辛%
%
氏有才子八人伯奮仲堪叔%
獻季仲伯虎仲熊叔豹季貍%
]%
此十六族者丗濟其美%
|[%
索隱曰謂元愷各有親族故稱%
族也濟成也言後代成前代也%
]%
不隕其名至於堯堯%
未能舉舜舉八愷使主后土|[%
王肅曰君治九土之冝%
杜預曰后土地也。索%
%
隱曰禹爲司空司空主土則禹在八愷之中。正%
義曰春秋正義云后君也天曰皇天地曰后土%
]%
以揆百%
事莫不時序|[%
正義曰言禹度九土之冝%
無不以時得其次序也%
]%
舉八元使布%
五教于四方|[%
索隱曰契爲司徒司徒敷%
五教則契在八元之數%
]%
父義母慈兄%
友弟恭子孝內平外成|[%
正義曰杜預云內諸夏外夷%
狄也按契作五常之教諸夏%
%
太平夷狄%
向化也%
]%
昔帝鴻氏有不才子|[%
賈逵曰帝鴻黄帝也%
不才子其苗裔讙兠%
%
也%
%
]%
掩義隱賊好行凶慝天下謂之渾沌|[%
正義曰慝%
惡也一本%
%
云天下之民謂之渾沌渾沌即讙兠也言掩義事隂爲賊害%
而好凶惡故謂之渾沌也杜預云渾沌不開通之皃神異經%
%
云崑崙西有獸焉其狀如犬長毛四足似羆而無爪有目而%
不見行不開有兩耳而不聞有人知往有腹無五臟有腹直%腸
%
短食徑過人有德行而往扺角有凶惡而行依憑之名渾沌%
又莊子云南海之帝爲儵忽中央之帝爲渾沌儵忽時相遇%
%
於渾沌之地渾沌待之甚善儵與忽謀欲報渾沌之德曰人%
皆有七竅以視聽食息此獨無有甞試鑿之日鑿一竅七日%
%
而渾沌死按言讙%
兠性似故號之也%
]%
少皡氏|[%
服䖍曰金%
天氏帝號%
]%
有不才子毀信%
惡忠崇飾惡言天下謂之窮竒|[%
服䖍曰謂共工氏%
也其行窮而好竒%
%
。正義曰謂共工言毀敗信行惡其忠直有惡言語髙粉飾%
之故謂之窮竒按常行終必窮極好謟䛕竒異於人也神異%諂
%
經云西北有獸其狀似虎有翼能飛便勦食人知人言語聞%
人鬬輙食直者聞人忠信輒食其鼻聞人惡逆不善輒殺獸%
%
往饋之名曰窮竒按言%
共工性似故號之也%
]%
顓頊氏有不才子不可教訓%
不知話言天下謂之檮杌|[%
賈逵曰檮杌頑凶無疇匹%
之貌謂鯀也。正義曰檮%
%
音道刀反杌音五骨反謂鮌也凶頑不可教訓不從詔令故%
謂之檮杌按言無疇疋言自縱恣也神異經云西方荒中有%
%
獸焉其狀如虎而大毛長二尺人面虎足豬口牙尾長一丈%
八尺攪亂荒中名檮杌一名傲佷一名難訓按言鯀性似故%
%
號之%
也%
]%
此三族丗憂之至于堯堯未能去縉雲氏%
|[%
賈逵曰縉雲氏姜姓也炎帝之苗裔當黄帝時在縉雲之官%
也。正義曰今括州縉雲縣蓋其所封也書云縉赤繒也%
]%
有不才子|[%
正義曰此以上四處皆左傳文或本有並文次%
相類四凶故書之恐本錯脫耳謂三苗也言貪%
%
飲食冒貨賄故謂之饕餮神異經云西南有人焉身多毛上%
頭戴豕性佷惡好息積財而不用善奪人穀物強者畏羣而%
%
單名饕餮言三%
苗性似故號之%
]%
貪于飲食冒于貨賄天下謂之饕%
餮天下惡之比之三凶|[%
杜預曰非帝子孫故%
別之以比三凶也%
]%
舜賔%
於四門|[%
正義曰杜預云闢四門%
達四聦以賔衆賢之也%
]%
乃流四凶族遷于四%
裔|[%
賈逵曰四裔之地%
去王城四千里%
]%
以御螭魅|[%
服䖍曰螭魅人面獸身四%
足好惑人山林異氣所生%
%
以爲人害。正義曰御音魚呂反螭音丑知反魅音媚按御%
魑魅恐更有邪謟之人故流放四凶以禦之也故下云無凶%諂
%
人%
也%
]%
於是四門辟言毋凶人也舜入于大麓烈風%
雷雨不迷堯乃知舜之足授天下堯老使舜攝%
行天子政廵狩舜得舉用事二十年而堯使攝%
政攝政八年而堯崩三年䘮畢讓丹朱天下歸%
舜而禹臯陶契后稷伯夷夔龍垂益彭祖|[%
索隱%
曰彭%
%
祖即陸終氏之第三子籛鏗之後後爲大彭亦稱彭祖。正%
義曰髙姚二音臯陶字庭堅英六二國是其後也契音薛殷%
%
之祖也伯夷齊太公之祖也夔巨龜反樂官也倕音垂亦作%
垂內言之官也益伯翳也即秦趙之祖彭祖自堯時舉用歷%
%
夏殷封%
於大彭%
]%
自堯時而皆舉用未有分職|[%
正義曰分音%
符問反如字%
%
分謂封疆%
爵土也%
]%
於是舜乃至於文祖謀于四嶽辟四%
門明通四方耳目命十二牧論帝德行厚德逺%
佞人|[%
正義曰舜命十二牧論帝堯之德又敦之於民%
逺離邪佞之人言能如此則夷狄亦服從也%
]%
則蠻%
夷率服舜謂四嶽曰有能奮庸|[%
馬融曰奮%
明庸功也%
]%
美堯%
之事者使居官相事皆曰伯禹爲司空可美帝%
功舜曰嗟然禹汝平水土維是勉哉禹拜稽首%
讓於稷契與臯陶舜曰然徃矣|[%
鄭玄曰然其舉得%
其人汝往居此官%
%
不聽其%
所讓也%
]%
舜曰棄黎民始飢|[%
徐廣曰今文尚書作祖飢祖%
始也。索隱曰古文作阻飢%
%
孔氏以爲阻難也祖%
阻聲相近未知誰得%
]%
汝后稷播時百穀|[%
鄭玄曰時讀曰%
蒔。正義曰稷%
%
農官也播時謂順%
四時而種百穀%
]%
舜曰契百姓不親五品不馴|[%
鄭%
玄%
%
曰五品父母兄弟子也王肅曰%
五品五常也。正義曰馴音訓%
]%
汝爲司徒而敬敷五教%
在寛|[%
馬融曰五%
品之教%
]%
舜曰臯陶蠻夷猾夏|[%
鄭玄曰猾夏%
侵亂中國也%
]%
寇賊姦軌|[%
鄭玄曰由內爲姦起外%
爲軌。正義亦作宄%
]%
汝作士|[%
馬融曰獄官%
之長。正義%
%
曰按若大%
理卿也%
]%
五刑有服|[%
正義曰孔安國云服從也言輕重之%
中正也按墨點鑿其額涅以墨劓截%
%
鼻也剕刖足也宮淫刑也男子%
割勢婦人幽閉也大辟死刑也%
]%
五服三就|[%
馬融曰五刑墨%
劓剕宮大辟三%
%
就謂大罪陳諸原野次罪於市朝同%
族適甸師氏旣服五刑當就三處%
]%
五流有度|[%
正義曰度%
音徒洛反%
%
尚書作宅孔安國云五%
刑之流各有所居也%
]%
五度三居|[%
正義曰按謂度其逺%
近爲三等之居也%
]%
維%
明能信|[%
馬融曰謂在八議君不忍刑宥之以逺五等之差%
亦有三等之居大罪投四裔次九州之外次中國%
%
之外當明其罪%
能使信服之%
]%
舜曰誰能馴予工|[%
馬融曰謂主%
百工之官也%
]%
皆曰%
垂可於是以垂爲共工|[%
馬融曰爲司空%
共理百工之事%
]%
舜曰誰能%
馴予上下|[%
馬融曰上謂%
原下謂隰%
]%
草木鳥獸皆曰益可於是%
以益爲朕虞|[%
馬融曰虞掌%
山澤之官名%
]%
益拜稽首讓于諸臣%
朱虎熊羆|[%
索隱曰即髙辛氏之子伯虎仲熊也。正義%
曰孔安國云朱虎熊羆二臣也垂益所讓四%
%
人皆在元%
凱之中也%
]%
舜曰徃矣汝諧遂以朱虎熊羆爲佐|[%
正%
義%
%
曰爲益%
之佐也%
]%
舜曰嗟四嶽有能典朕三禮|[%
馬融曰三禮天%
神地祇人鬼之%
%
禮也鄭玄曰天事%
地事人事之禮也%
]%
皆曰伯夷可舜曰嗟伯夷以汝%
爲秩宗|[%
鄭玄曰主次秩尊卑。正義曰若太常也漢書百%
官表云王莽太常曰秩宗依古也孔安國云秩序%
%
宗尊也主郊%
廟之官也%
]%
夙夜維敬直哉維靜㓗|[%
正義曰靜清也%
㓗明也孔安國%
%
云職典禮施政教%
使正直而清明%
]%
伯夷讓夔龍舜曰然|[%
正義曰孔安%
國云然其推%
%
賢不許%
其讓也%
]%
以夔爲典樂教稺子|[%
鄭玄曰國子也案尚書%
作冑子孔安國曰稺冑%
%
聲相近。正義曰稺冑雉反孔安國云冑長子謂元子以下%
至卿大夫子弟也歌詩蹈之舞之教長國子中和祗庸孝友%
]%
直而溫|[%
馬融曰正直%
而色溫和%
]%
寛而栗|[%
馬融曰寛大而%
謹敬戰栗也%
]%
剛而無%
虐簡而毋傲|[%
正義曰孔安國云剛失之虐%
簡失之傲教之以防其失也%
]%
詩言意歌%
長言|[%
馬融曰歌所以長言詩之意也。正義曰孔安%
國云詩言志以蹈其心歌詠其義以長其言也%
]%
聲依%
永律和聲|[%
鄭玄曰聲之曲折又依長言聲中律乃爲和也%
。正義曰孔安國云聲五聲宮商角徴羽也律%
%
謂六律六呂十二月之音%
氣也當依聲律和樂也%
]%
八音能諧毋相奪倫神人%
以和|[%
鄭玄曰祖考來格羣后德讓其一隅也。正義曰八%
音金石絲竹匏土革木也孔安國云倫理也八音能%
%
諧理不錯奪則神人%
咸和命夔使勉也%
]%
夔曰於予擊石拊石百獸率%
舞|[%
鄭玄曰百獸服不氏所養者也率舞言音和也。正義曰%
於音烏孔安國云石磬音之清者拊亦擊也舉清者和則%
%
其餘皆從矣樂感百獸使相率而舞則神人和可知也按磬%
一片黑石也下音福尤反周禮云夏官有服不氏掌服猛獸%
%
下士一人徒四人鄭%
玄云不服之獸也%
]%
舜曰龍朕畏忌讒說殄僞振%
驚朕衆|[%
徐廣曰一云齊說殄行振驚衆駰案鄭玄曰所謂%
色取仁而行違是驚動我之衆臣使之疑惑。正%
%
義曰僞音危腫反言畏惡利口讒說之人兼殄絕姦僞人黨%
恐其驚動我衆使龍遏絕之岀入其命惟信實也此僞字太%
%
史公變尚書文也尚書僞字作行音下孟反言%
言畏忌有利口讒說之人殄絕無德行之官也%
]%
命汝爲納%
言夙夜岀入朕命惟信|[%
正義曰孔安國云納言喉舌%
之官也聽下言納於上受上%
%
言宣於下%
必信也%
]%
舜曰嗟女二十有二人|[%
馬融曰稷契臯陶皆%
居官乆有成功但述%
%
而美之無所復勑禹及垂已下皆初命凢六人與上十二%
牧四嶽凢二十二人鄭玄曰皆格于文祖時所勑命也%
]%
敬%
哉惟時相天事|[%
正義曰相視也舜命二十二人各敬行%
其職惟在順時視天所冝而行事也%
]%
三歳一考功三考絀陟逺近衆功咸興分北三%
苗|[%
鄭玄曰所竄三苗爲西裔諸%
侯者猶爲惡乃復分析流之%
]%
此二十二人咸成厥功%
臯陶爲大理平|[%
正義曰臯陶作士%
正平天下罪惡也%
]%
民各伏得其實%
伯夷主禮上下咸讓垂主工師|[%
正義曰工師若%
今大匠卿也%
]%
百%
工致功益主虞山澤辟|[%
正義曰婢%
亦反開也%
]%
棄主稷百穀%
時茂契主司徒百姓親和龍主賔客逺人至十%
二牧行而九州莫敢辟違|[%
正義曰禹九州之民無%
敢辟違舜十二牧也%
]%
唯%
禹之功爲大披九山|[%
正義曰披音皮義反%
謂傍其山邊以通%
]%
通九澤決%
九河定九州各以其職來貢不失厥冝方五千%
里至于荒服南撫交阯北發|[%
索隱曰%
一句%
]%
西戎析枝渠%
廋氐羌|[%
索隱曰%
一句%
]%
北山戎發息愼|[%
鄭玄曰息愼或謂%
之肅愼東北夷%
]%
東%
長鳥夷|[%
索隱曰此言帝舜之德皆撫及四方夷人故先以%
撫字揔之北發當云北戶南方有地名北戶又案%
%
漢書北發是北方國名今以北發爲南方之國誤也此文省%
略四夷之名錯亂西戎上少一西字山戎下少一北字長字%
%
下少一夷字長夷也鳥夷也其意冝然今案大戴禮亦云長%
夷則長是夷號又云鮮攴渠搜則鮮支當此析枝也鮮析音%支
%
相近鄒氏劉氏云息並音肅非也且夷狄之名古書不必皆%
同今讀如字也。正義曰注鳥或作島括地志云百濟國西%
%
南海中有大島十五所皆置邑有人居屬百濟又倭國西南%
大海中島居凡百餘小國在京南萬三千五百里按武后改%
%
倭國爲%
日本國%
]%
四海之內|[%
正義曰爾雅云九夷八%
狄七戎六蠻謂之四海%
]%
咸戴帝舜之%
功於是禹乃興九招之樂|[%
索隱曰招音韶即舜樂%
簫韶九成故曰九招%
]%
致%
異物鳳皇來翔天下明德皆自虞帝始舜年二%
十以孝聞年三十堯舉之年五十攝行天子事%
年五十八堯崩年六十一代堯踐帝位|[%
皇甫謐曰%
舜所都或%
%
言蒲阪或言平陽或言潘潘今上谷也。正義曰括地志云%
平陽今晉州城是也潘今嬀州城是也蒲坂今蒲州南二里%
%
河東縣界蒲%
坂故城是也%
]%
踐帝位三十九年南廵狩崩於蒼梧%
之野葬於江南九疑是爲零陵|[%
皇覽曰舜冢在零%
陵營浦縣其山九%
%
谿皆相似故曰九疑傳曰舜葬蒼梧象爲之耕禮記曰舜葬%
蒼梧二妃不從山海經曰蒼梧山帝舜葬于陽丹朱葬于隂%
%
皇甫謐曰或曰%
二妃葬衡山%
]%
舜之踐帝位載天子旗往朝父瞽%
叟夔夔唯謹|[%
徐廣曰%
和敬貌%
]%
如子道封弟象爲諸侯|[%
孟%
子%
%
曰封之有庳音鼻。正義曰帝王紀云舜弟象封於有鼻括%
地志云鼻亭神在道縣北六十里故老傳云舜葬九疑象來%
%
至此後人立祠名爲鼻亭神輿地志云零陵郡應陽縣東有%
山山有象廟王隱晉書云此大泉陵縣北部東五里有鼻墟%
%
象所%
封也%
]%
舜子商均亦不肖|[%
皇甫謐曰娥皇無子女英生商%
均。正義曰譙周云以虞封舜%
%
子今宋州虞城縣括地志云虞國舜後所%
封邑也或云封舜子均於商故號商均也%
]%
舜乃豫薦禹%
於天|[%
索隱曰謂告天%
使之攝位也%
]%
十七年而崩三年䘮畢禹亦%
乃讓舜子|[%
正義曰括地志云禹居洛州%
陽城者避商均非時乆居也%
]%
如舜讓堯子%
諸侯歸之然後禹踐天子位堯子丹朱舜子商%
均皆有疆土|[%
譙周曰以唐封堯之子以虞封舜之子。索%
隱曰漢書律曆志云封堯子朱於丹淵爲諸%
%
侯商均封虞在梁國今虞城縣也。正義曰括地志%
云定州唐縣堯後所封宋州虞城縣舜後所封也%
]%
以奉%
先祀服其服禮樂如之以客見天子|[%
正義曰爲天%
子之賔客也%
]%
天子弗臣示不敢專也自黄帝至舜禹皆同姓%
而異其國號以章明德|[%
徐廣曰外傳曰黄帝二十五%
子其得姓者十四人虞翻云%
%
以德爲氏姓又虞說以凢有二十五人其二人同姓姬又十%
一人爲十一姓酉祁已滕葴任荀釐姞嬛衣是也餘十二姓%
%
德薄不%
紀録%
]%
故黄帝爲有熊帝顓頊爲髙陽帝嚳爲%
髙辛帝堯爲陶唐|[%
韋昭曰陶唐皆國名猶湯稱殷商%
矣張晏曰堯爲唐侯國於中山唐%
%
縣是%
也%
]%
帝舜爲有虞|[%
皇甫謐曰舜𡣕于虞因以爲氏%
今河東大陽西山上虞城是也%
]%
帝%
禹爲夏后而別氏姓姒氏契爲商姓子氏|[%
索隱%
曰禮%
%
緯曰禹母脩己吞薏苡而生禹因姓姒氏%
而契姓子氏者亦以其母吞乙子而生%
]%
棄爲周姓姬%
氏|[%
鄭玄駮許愼五經異義曰春秋左傳無駭卒羽父請謚與%
族公問族於衆仲衆仲對曰天子建德因生以賜姓胙之%
%
土而命之氏諸侯以字爲氏因以爲族官有丗功則有官族邑%
亦如之公命以字爲展氏以此言之天子賜姓命氏諸侯命%
%
族族者氏之別名也姓者所以統繫百丗使不別也氏者所%
以別子孫之所岀故丗本之篇言姓則在上言氏則在下也%
]%
