虞舜者|[%
謚法曰仁聖盛明曰舜。索隱曰虞國名在河東%
太陽縣舜謚也皇甫謐云舜字都君也。正義曰%
%
括地志云故虞城在陝州河北縣東北五十里虞山之上酈%
元注水經云幹橋東北有虞城堯以女𡣕于虞之地也又宋州虞城大襄國所封之邑杜預云舜後諸侯也又越州餘姚縣顧野王云舜後支庶所封之地舜姚姓故云餘姚縣西七十里有漢上虞故縣會稽舊記云舜上虞人去虞三十里有姚丘即舜所生也周處風土記云舜東夷之人生姚丘括地志又云姚墟在濮州雷澤縣東十三里孝經援神契云舜生於姚墟案二
32
所未詳也名曰重華|[%
二]徐廣曰﹕皇甫謐云舜以堯之二十一年甲子生三十一年甲午徵用七十九年壬午即真百歳癸卯崩正義曰尚書云﹕重華協於帝孔安國云華謂文德也言其光文重合於堯瞽叟姓媯妻曰握登見大虹意感而生舜於姚墟故姓姚目重瞳子故曰重華字都君龍顏大口黑色身長六尺一寸重華父曰瞽叟|[%
三]正義曰先后反孔安國云﹕無目曰瞽舜父有目不能分別好惡故時人謂之瞽配字曰叟叟無目之稱也瞽叟父曰橋牛|[%
四]正義曰橋又音嬌橋牛父曰句望|[%
五]正義曰句古侯反望音亡句望父曰敬康敬康父曰窮蟬窮蟬父曰帝顓頊顓頊父曰昌意以至舜七世矣自從窮蟬以至帝舜皆微爲庶人

舜父瞽叟盲而舜母|[%
一]索隱曰皇甫謐云舜母名握登生舜於姚墟因姓姚氏也死瞽叟更娶妻而生象象傲瞽叟愛後妻子常欲殺舜舜避逃;及有小過則受罪順事父及後母與弟日以篤謹匪有解

舜冀州之人也|[%
一]正義曰蒲州河東縣本屬冀州宋永初山川記云蒲坂城中有舜廟城外有舜宅及二妃檀括地志云媯州有媯水源出城中耆舊傳云即舜釐降二女於媯汭之所外城中有舜井城北有歷山山上有舜廟未詳案媯州亦冀州城是也舜耕歷山|[%
二]鄭玄曰在河東正義曰括地志云蒲州河東縣雷首山一名中條山亦名歷山亦名首陽山亦名蒲山亦名襄山亦名甘棗山亦名猪山亦名狗頭山亦名薄山亦名吳山此山西起雷首山東至吳坂凡十一名隨州縣分之歷山南有舜井又云越州餘姚縣有歷山舜井濮州雷澤縣有歷山舜井二所又有姚墟云生舜處也及媯州歷山舜井皆云舜所耕處未詳也漁雷澤|[%
三]鄭玄曰雷夏兗州澤今屬濟陰正義曰括地志云雷夏澤在濮州雷澤縣郭外西北山海經云雷澤有雷神龍身人頭鼓其腹則雷也陶河濱|[%
四]皇甫謐曰濟陰定陶西南陶丘亭是也正義曰案於曹州濱河作瓦器也括地志云陶城在蒲州河東縣北三十里即舜所都也南去歷山不遠或耕或陶所在則可何必定陶方得爲陶也舜之陶也斯或一焉作什器於壽丘|[%
五]皇甫謐曰在魯東門之北索隱曰什器什數也蓋人家常用之器非一故以十爲數猶今云什物也壽丘地名黃帝生處正義曰壽音受顏師古云軍法伍人爲伍二伍爲什則共器物故謂生生之具爲什器亦猶從軍及作役者十人爲火共畜調度也就時於負夏|[%
六]鄭玄曰負夏衞地索隱曰就時猶逐時若言乘時射利也尚書大傳曰販於頓丘就時負夏孟子曰遷于負夏是也舜父瞽叟頑母嚚弟象傲皆欲殺舜舜順適不失子道兄弟孝慈欲殺不可得;即求嘗在側

33
舜年二十以孝聞三十而帝堯問可用者|[%
一]正義曰可用謂可爲天子也四嶽咸薦虞舜曰可於是堯乃以二女妻舜以觀其內使九男與處以觀其外舜居媯汭內行彌謹堯二女不敢以貴驕事舜親戚|[%
二]正義曰二女不敢以帝女驕慢舜之親戚親戚謂父瞽叟後母弟象妹顆手等也顆音苦果反甚有婦道堯九男皆益篤|[%
三]正義曰篤惇也非唯二女恭勤婦道九男事舜皆益惇厚謹敬也舜耕歷山歷山之人皆讓畔;|[%
四]正義曰韓非子歷山之農相侵略舜往耕朞年耕者讓畔也漁雷澤雷澤上人
34
皆讓居;陶河濱河濱器皆不苦窳|[%
五]史記音隱曰音游甫反駰謂寙病也正義曰苦讀如盬音古盬麤也寙音庾一年而所居成聚|[%
六]正義曰聚在喻反謂村落也二年成邑三年成都|[%
七]正義曰周禮郊野法云九夫爲井四井爲邑四邑爲丘四丘爲甸四甸爲縣四縣爲都也堯乃賜舜絺衣|[%
八]正義曰絺勑遲反細葛布衣也鄒氏音竹几反與琴爲築倉廩予牛羊瞽叟尚復欲殺之使舜上塗廩瞽叟從下縱火焚廩舜乃以兩笠自扞而下去得不死|[%
九]索隱曰言以笠自扞己身有似鳥張翅而輕下得不損傷皇甫謐云兩繖繖笠類列女傳云二女教舜鳥工上廩是也正義曰通史云瞽叟使舜滌廩舜告堯二女女曰時其焚汝鵲汝衣裳鳥工往舜旣登廩得免去也後瞽叟又使舜穿井舜穿井爲匿空|[%
一0]索隱曰音孔列女傳所謂龍工入井是也旁出|[%
一一]正義曰言舜潛匿穿孔旁從他井而出也通史云舜穿井又告二女二女曰去汝裳衣龍工往入井瞽叟與象下土實井舜從他井出去也括地志云舜井在媯州懷戎縣西外城中其西又有一井耆舊傳云並舜井也舜自中出帝王紀云河東有舜井未詳也舜旣入深瞽叟與象共下土實井|[%
一二]索隱曰亦作填井舜從匿空出去|[%
一三]劉熙曰舜以權謀自免亦大聖有神人之助也瞽叟象喜以舜爲已死象曰本謀者象象與其父母分|[%
一四]正義曰扶問反於是曰舜妻堯二女與琴象取之牛羊倉廩予父母象乃止舜宮居|[%
一五]正義曰宮即室也爾雅云室謂之宮禮云命士已上父子異宮也鼓其琴舜往見之象鄂不懌曰我思舜正鬱陶舜曰然爾其庶矣|[%
一六]索隱曰言汝猶當庶幾於友悌之情義也如孟子取尚書文又云惟茲臣庶女其于予治蓋欲令象共我理臣庶也舜復事瞽叟愛弟彌謹於是堯乃試舜五典百官皆治

35
昔高陽氏有才子八人|[%
一]名見左傳世得其利謂之八愷|[%
二]賈逵曰愷和也索隱曰左傳史克對魯宣公曰昔高陽氏有才子八人倉舒隤皚檮戭大臨尨降庭堅仲容叔達高辛氏有才子八人|[%
三]名見左傳世謂之八元|[%
四]賈逵曰元善也索隱曰左傳高辛氏有才子八人伯奮仲堪叔獻季仲伯虎仲熊叔豹季貍此十六族者世濟其美|[%
五]索隱曰謂元愷各有親族故稱族也濟成也言後代成前代也不隕其名至於堯堯未能舉舜舉八愷使主后土|[%
六]王肅曰君治九土之宜杜預曰后土地官索隱曰主土禹爲司空司空主土則禹在八愷之中正義曰春秋正義云后君也天曰皇天地曰后土以揆百事莫不時序|[%
七]正義曰言禹度九土之宜無不以時得其次序也舉八元使布五教于四方|[%
八]索隱曰契爲司徒司徒敷五教則契在八元之數父義母慈兄友弟恭子孝內平外成|[%
九]正義曰杜預云內諸夏外夷狄也案契作五常之教諸夏太平夷狄向化也

36
昔帝鴻氏有不才子|[%
一]賈逵曰帝鴻黃帝也不才子其苗裔讙兜也掩義隱賊好行凶慝天下謂之渾沌|[%
二]正義曰慝惡也一本云天下之民謂之渾沌渾沌即讙兜也言掩義事陰爲賊害而好凶惡故謂之渾沌也杜預云渾沌不開通之貌神異經云崑崙西有獸焉其狀如犬長毛四足似羆而無爪有目而不見行不開有兩耳而不聞有人知性有腹無五藏有腸直而不旋食徑過人有德行而往扺觸之有凶德則往依憑之名渾沌又莊子云南海之帝爲儵北海之帝爲忽中央之帝爲渾沌儵忽乃相遇於渾沌之地渾沌待之甚善儵與忽謀欲報渾沌之德曰人皆有七竅以視聽食息此獨無有嘗試鑿之日鑿一竅七日而渾沌死案言讙兜性似故號之也少暤氏|[%
三]服虔曰金天氏帝號有不才子毀信惡忠崇飾惡言天下謂之窮奇|[%
四]服虔曰謂共工氏也其行窮而好奇正義曰謂共工言毀敗信行惡其忠直有惡言語高粉飾之故謂之窮奇案常行終必窮極好諂諛奇異於人也神異經云西北有獸其狀似虎有翼能飛便勦食人知人言語聞人鬬輒食直者聞人忠信輒食其鼻聞人惡逆不善輒殺獸往饋之名曰窮奇案言共工性似故號之也顓頊氏有不才子不可教訓不知話言天下謂之檮杌|[%
五]賈逵曰檮杌頑凶無疇匹之貌謂鯀也正義曰檮音道刀反杌音五骨反謂鯀也凶頑不可教訓不從詔令故謂之檮杌案言無疇匹言自縱恣也神異經云西方荒中有獸焉其狀如虎而大毛長二尺人面虎足豬口牙尾長一丈八尺攪亂荒中名檮杌一名傲很一名難訓案言鯀性似故號之也此三族世憂之至于堯堯未能去縉雲氏|[%
六]賈逵曰縉雲氏姜姓也炎帝之苗裔當黃帝時任縉雲之官也正義曰今括州縉雲縣蓋其所封也字書云縉赤繒也有不才子貪于飲食冒于貨賄天下謂之饕餮|[%
七]正義曰謂三苗也言貪飲食冒貨賄故謂之饕餮神異經云西南有人焉身多毛頭上戴豕性很惡好息積財而不用善奪人穀物強者奪老弱者畏羣而擊單名饕餮言三苗性似故號之天下惡之比之三凶|[%
八]杜預曰非帝子孫故別之以比三凶也正義曰此以上四處皆左傳文或本有並文次相類四凶故書之恐本錯脫耳舜賓於四門|[%
九]正義曰杜預云闢四門達四聰以賓禮眾賢也乃流四凶族遷于四裔|[%
一0]賈逵曰四裔之地去王城四千里以
37
御螭魅|[%
一一]服虔曰螭魅人面獸身四足好惑人山林異氣所生以爲人害正義曰御音魚呂反螭音丑知反魅音媚案御魑魅恐更有邪諂之人故流放四凶以禦之也故下云無凶人也於是四門辟言毋凶人也

38
舜入于大麓烈風雷雨不迷堯乃知舜之足授天下堯老使舜攝行天子政巡狩舜得舉用事二十年而堯使攝政攝政八年而堯崩三年喪畢讓丹朱天下歸舜而禹皋陶|[%
一]正義曰高姚二音契后稷伯夷夔龍倕益彭祖|[%
二]索隱曰彭祖即陸終氏之第三子籛鏗之後後爲大彭亦稱彭祖正義曰皋陶字庭堅英六二國是其後也契音薛殷之祖也伯夷齊太公之祖也夔巨龜反樂官也倕音垂亦作垂內言之官也益伯翳也即秦趙之祖彭祖自堯時舉用歷夏殷封於大彭自堯時而皆舉用未有分職|[%
三]正義曰分音符問反又如字分謂封疆爵土也於是舜乃至於文祖謀于四嶽辟四門明通四方耳目命十二牧論帝德行厚德遠佞人|[%
四]正義曰舜命十二牧論帝堯之德又敦之於民遠離邪佞之人言能如此則夷狄亦服從也則蠻夷率服舜謂四嶽曰有能奮庸|[%
五]馬融曰奮明;庸功也美堯之事者使居官相事皆曰伯禹爲司空可美帝功舜曰嗟然禹汝平水土維是勉哉禹拜稽首讓於稷契與皋陶舜曰然往矣|[%
六]鄭玄曰然其舉得其人汝往居此官不聽其所讓也舜曰弃黎民始飢|[%
七]徐廣曰今文尚書作祖飢祖始也索隱曰古文作阻飢孔氏以爲阻難也祖阻聲相近未知誰得汝后稷播時百穀|[%
八]鄭玄曰時讀曰蒔正義曰稷農官也播時謂順四時而種百穀舜曰契百姓不親五
39
品不馴|[%
九]鄭玄曰五品父母兄弟子也王肅曰五品五常也正義曰馴音訓汝爲司徒而敬敷五教在寬|[%
一0]馬融曰五品之教舜曰皋陶蠻夷猾夏|[%
一一]鄭玄曰猾夏侵亂中國也寇賊姦軌|[%
一二]鄭玄曰由內爲姦起外爲軌正義曰亦作宄汝作士|[%
一三]馬融曰獄官之長正義曰案若大理卿也五刑有服五服三就;|[%
一四]馬融曰五刑墨劓剕宮大辟三就謂大罪陳諸原野次罪於市朝同族適甸師氏旣服五刑當就三處正義曰孔安國云服從也言得輕重之中正也案墨點鑿其額涅以墨劓截鼻也剕刖足也宮淫刑也男子割勢婦人幽閉也大辟死刑也五流有度|[%
一五]正義曰度音徒洛反尚書作宅孔安國云五刑之流各有所居也五度三居|[%
一六]正義曰案謂度其遠近爲三等之居也維明能信|[%
一七]馬融曰謂在八議君不忍刑宥之以遠五等之差亦有三等之居大罪投四裔次九州之外次中國之外當明其罪能使信服之舜曰誰能馴予工|[%
一八]馬融曰謂主百工之官也皆曰垂可於是以垂爲共工|[%
一九]馬融曰爲司空共理百工之事舜曰誰能馴予上下|[%
二0]馬融曰上謂原下謂隰草木鳥獸皆曰益可於是以益爲朕虞|[%
二一]馬融曰虞掌山澤之官名益拜稽首讓于諸臣朱虎熊羆|[%
二二]索隱曰即高辛氏之子伯虎仲熊也正義曰孔安國云朱虎熊羆二臣名垂益所讓四人皆在元凱之中也舜曰往矣汝諧遂以朱虎熊羆爲佐|[%
二三]正義曰爲益之佐也舜曰嗟四嶽有能典朕三禮|[%
二四]馬融曰三禮天神地祇人鬼之禮也鄭玄曰天事地事人事之禮皆曰伯夷可舜曰嗟伯夷以汝爲秩宗|[%
二五]鄭玄曰主次秩尊卑正義曰若太常也漢書百官表云王莽改太常曰秩宗依古也孔安國云秩序;宗尊也主郊廟之官也夙夜維敬直哉維靜絜|[%
二六]正義曰靜清也絜明也孔安國云職典禮施政教使正直而清明伯夷讓夔龍舜曰然|[%
二七]正義曰孔安國云然其推賢不許其讓也以夔爲典樂教稺子|[%
二八]鄭玄曰國子也案尚書作冑子稺冑聲相近正義曰稺冑雉反孔安國云冑長也謂元子以下至卿大夫子弟以歌詩蹈之舞之教長國子中和祗庸孝友直而溫|[%
二九]馬融曰正直而色溫和寬而栗|[%
三0]馬融曰寬大而謹敬戰栗也剛而毋虐簡而毋傲;|[%
三一]正義曰孔安國云剛失之虐簡失之傲教之以防其失也詩言意歌長言|[%
三二]馬融曰歌所以長言詩之意也正義曰孔安國云詩言志以導其心歌詠其義以長其言也聲依永律和聲|[%
三三]鄭玄曰聲之曲折又依長言聲中律乃爲和也正義曰孔安國云聲五聲宮商角徵羽也律謂六律六呂十二月之音氣也當依聲律和樂也八音能諧毋相奪倫神人以和|[%
三四]鄭玄曰祖考來格羣后德讓其一隅也正義曰八音金石絲竹匏土革木也孔安國云倫理也八音能諧理不錯奪則神人咸和命夔使勉也夔曰於予擊石拊石百獸率舞|[%
三五]鄭玄曰百獸服不氏所養者也率舞言音和也正義曰於音烏孔安國云石磬音之清者拊亦擊也舉清者和則其餘皆從矣樂感百獸使相率而舞則神人和可知也案磬一片黑石也不音福尤反周禮云夏官有服不氏掌服猛獸下士一人徒四人鄭玄云服不服之獸也舜曰龍朕畏忌讒說殄偽振驚朕眾|[%
三六]徐廣曰一云齊說殄行振驚眾駰案鄭玄曰所謂色取仁而行違是驚動我之眾臣使之疑惑正義曰偽音危睡反言畏惡利口讒說之人兼殄絕姦偽人黨恐其驚動我眾使龍遏絕之出入其命惟信實也此偽字太史公變尚書文也尚書偽字作行音下孟反言己畏忌有利口讒說之人殄絕無德行之官也命汝爲納言夙夜出入朕命惟信|[%
三七]正義曰孔安國云納言喉舌之官也聽下言納於上受上言宣於下必信也舜曰嗟女二十有二人|[%
三八]馬融曰稷契皋陶皆居官久有成功但述而美之無所復勅禹及垂已下皆初命凡六人與上十二牧四嶽凡二十二人鄭玄曰皆格于文祖時所勅命也敬哉惟時相天事|[%
三九]正義曰相視也舜命二十二人各敬行其職惟在順時視天所宜而行事也三歳一考功三考絀陟遠近眾功咸興分北三苗|[%
四0]鄭玄曰所竄三苗爲西裔諸侯者猶爲惡乃復分析流之

43
此二十二人咸成厥功皋陶爲大理平|[%
一]正義曰皋陶作士正平天下罪惡也民各伏得其實;伯夷主禮上下咸讓;垂主工師|[%
二]正義曰工師若今大匠卿也百工致功;益主虞山澤辟;|[%
三]正義曰婢亦反開也弃主稷百穀時茂;契主司徒百姓親和;龍主賓客遠人至;十二牧行而九州莫敢辟違;|[%
四]正義曰禹九州之民無敢辟違舜十二牧也唯禹之功爲大披九山|[%
五]正義曰披音皮義反謂傍其山邊以通通九澤決九河定九州各以其職來貢不失厥宜方五千里至于荒服南撫交阯北發|[%
六]索隱曰一句西戎析枝渠廋氐羌|[%
七]索隱曰一句北山戎發息慎|[%
八]鄭玄曰息慎或謂之肅慎東北夷東長鳥夷|[%
九]索隱曰此言帝舜之德皆撫及四方夷人故先以撫字總之北發當云北戶南方有地名北戶又案漢書北發是北方國名今以北發爲南方之國誤也此文省略四夷之名錯亂西戎上少一西字山戎下少一北字長字下少一夷字長夷也鳥夷也其意宜然今案大戴禮亦云長夷則長是夷號又云鮮支渠搜則鮮支當此析枝也鮮析音相近鄒氏劉氏云息並音肅非也且夷狄之名古書不必皆同今讀如字也正義曰注鳥或作島括地志云百濟國西南海中有大島十五所皆置邑有人居屬百濟又倭國西南大海中島居凡百餘小國在京南萬三千五百里案武后改倭國爲日本國四海之內|[%
一0]正義曰爾雅云九夷八狄七戎六蠻謂之四海咸戴帝舜之功於是禹乃興九招之樂|[%
一一]索隱曰招音韶即舜樂簫韶九成故曰九招致異物鳳皇來翔天下明德皆自虞帝始

44
舜年二十以孝聞年三十堯舉之年五十攝行天子事年五十八堯崩年六十一代堯踐帝位|[%
一]皇甫謐曰舜所都或言蒲阪或言平陽或言潘潘今上谷也正義曰括地志云平陽今晉州城是也潘今媯州城是也蒲阪今蒲州南二里河東縣界蒲阪故城是也踐帝位三十九年南巡狩崩於蒼梧之野葬於江南九疑是爲零陵|[%
二]皇覽曰舜冢在零陵營浦縣其山九谿皆相似故曰九疑傳曰舜葬蒼梧象爲之耕禮記曰舜葬蒼梧二妃不從山海經曰蒼梧山帝舜葬于陽丹朱葬于陰皇甫謐曰或曰二妃葬衡山舜之踐帝位載天子旗往朝父瞽叟夔夔唯謹|[%
三]徐廣曰和敬貌如子道封弟象爲諸侯|[%
四]孟子曰封之有庳音鼻正義曰帝王紀云舜弟象封於有鼻括地志云鼻亭神在營道縣北六十里故老傳云舜葬九疑象來至此後人立祠名爲鼻亭神輿地志云零陵郡應陽縣東有山山有象廟王隱晉書云本泉陵縣北部東五里有鼻墟象所封也舜子商均亦不肖|[%
五]皇甫謐曰娥皇無子女英生商均正義曰譙周云以虞封舜子今宋州虞城縣括地志云虞國舜後所封邑也或云封舜子均於商故號商均也舜乃豫薦禹於天|[%
六]索隱曰謂告天使之攝位也十七年而崩三年喪畢禹亦乃讓舜子|[%
七]正義曰括地志云禹居洛州陽城者避商均非時久居也如舜讓堯子諸侯歸之然後禹踐天子位堯子丹朱舜子商均皆有疆土|[%
八]譙周曰以唐封堯之子以虞封舜之子索隱曰漢書律曆志云封堯子朱於丹淵爲諸侯商均封虞在梁國今虞城縣也正義曰括地志云定州唐縣堯後所封宋州虞城縣舜後所封也以奉先祀服其服禮樂如之以客見天子|[%
九]正義曰爲天子之賓客也天子弗臣示不敢專也

45
自黃帝至舜禹皆同姓而異其國號以章明德|[%
一]徐廣曰外傳曰黃帝二十五子其得姓者十四人虞翻云以德爲氏姓又虞說以凡有二十五人其二人同姓姬又十一人爲十一姓酉祁已滕葴任荀釐姞儇衣是也餘十二姓德薄不紀錄正義曰釐音力其反姞音其吉反儇音在宣反故黃帝爲有熊帝顓頊爲高陽帝嚳爲高辛帝堯爲陶唐|[%
二]韋昭曰陶唐皆國名猶湯稱殷商矣張晏曰堯爲唐侯國於中山唐縣是也帝舜爲有虞|[%
三]皇甫謐曰舜𡣕于虞因以爲氏今河東大陽西山上虞城是也帝禹爲夏后而別氏姓姒氏契爲商姓子氏|[%
四]索隱曰禮緯曰禹母脩己吞薏苡而生禹因姓姒氏而契姓子氏者亦以其母吞乙子而生弃爲周姓姬氏|[%
五]鄭玄駮許慎五經異義曰春秋左傳無駭卒羽父請謚與族公問族於眾仲眾仲對曰天子建德因生以賜姓胙之土而命之氏諸侯以字爲氏因以爲族官有世功則有官族邑亦如之公命以字爲展氏以此言之天子賜姓命氏諸侯命族族者氏之別名也姓者所以統繫百世使不別也氏者所以別子孫之所出故世本之篇言姓則在上言氏則在下也

46
太史公曰|[%
一]正義曰太史公司馬遷自謂也自敍傳云太史公曰先人有言又云太史公曰余聞之董生又云太史公遭李陵之禍明太史公司馬遷自號也遷爲太史公官題贊首也虞憙云古者主天官者皆上公非獨遷學者多稱五帝尚矣|[%
二]索隱曰尚上也言久遠也然尚矣文出大戴禮然尚書獨載堯以來;而百家言黃帝其文不雅馴|[%
三]正義曰馴訓也謂百家之言皆非典雅之訓薦紳先生難言之|[%
四]徐廣曰薦紳即縉紳也古字假借孔子所傳宰予問五帝德及帝繫姓|[%
五]正義曰繫音奚計反儒者或不傳|[%
六]索隱曰五帝德帝繫姓皆大戴禮及孔子家語篇名以二者皆非正經故漢時儒者以爲非聖人之言故多不傳學也余嘗西至空桐|[%
七]正義曰余太史公自稱也嘗曾也空桐山在原州平高縣西百里黃帝問道於廣成子處北過涿鹿|[%
八]正義曰涿鹿山在媯州東南五十里山側有涿鹿城即黃帝堯舜之都也東漸於海南浮江淮矣至長老皆各往往稱黃帝堯舜之處風教固殊焉總之不離古文者近是|[%
九]索隱曰古文即帝德帝系二書也近是聖人之說予觀春秋國語其發明五帝德帝繫姓章矣|[%
一0]索隱曰太史公言己以春秋國語古書博加考驗益以發明五帝德等說甚章著也顧弟弗深考|[%
一一]徐廣曰弟但也史記漢書見此者非一又左思蜀都賦曰弟如滇池而不詳者多以爲字誤學者安可不博觀乎正義曰顧念也弟且也太史公言博考古文擇其言表見之不虛甚章著矣思念亦且不須更深考論其所表見皆不虛|[%
一二]索隱曰言帝德帝系所有表見者皆不爲虛妄也書缺有閒矣|[%
一三]正義曰言古文尚書缺失其閒多矣而無說黃帝之語其軼乃時時見於他說|[%
一四]索隱曰言古典殘缺有年載故曰有閒然皇帝遺事散軼乃時時旁見於他記說即帝德帝系等說也故己今採案而備論黃帝已來事耳非好學深思心知其意固難爲淺見寡聞道也余并論次擇其言尤雅者故著爲本紀書首|[%
一五]正義曰太史公據古文并諸子百家論次擇其言語典雅者故著爲五帝本紀在史記百三十篇書之首

48
【索隱述贊】帝出少典居于軒丘旣代炎曆遂禽蚩尤高陽嗣位靜深有謀小大遠近莫不懷柔爰洎帝嚳列聖同休帝摯之弟其號放勳就之如日望之如雲郁夷東作昧谷西曛明敭庂陋玄德升聞能讓天下賢哉二君
