太皡庖犧氏風姓代燧人氏繼天而王母曰華%
胥履大人迹於雷澤而生庖犧於成紀蛇身人%
首|[%
按伏犧風姓岀國語其華胥已下岀帝王丗紀然雷澤澤%
名即舜所漁之地在濟隂成紀亦地名按天水有成紀縣%
]%
有聖德仰則觀象於天俯則觀法於地旁觀鳥%
獸之文與地之冝近取諸身遠取諸物始畫八%
卦以通神明之德以類萬物之情造書契以代%
結繩之政於是始制嫁娶以儷皮爲禮|[%
按譙周古%
史考伏犧%
%
制嫁娶以儷%
皮爲禮也%
]%
結網罟以教佃漁故曰宓犧氏|[%
按事%
岀漢%
%
書歷志%
宓音伏%
]%
養犧牲以庖厨故曰庖犧有龍瑞以龍紀%
官號曰龍師作三十五弦之瑟木德王注春令%
故易稱帝岀乎震月令孟春其帝太皡是也|[%
按%
位%
%
在東方象日之明%
故稱太皡皡明也%
]%
都於陳東封太山立一十一年崩%
|[%
按皇甫謐伏犧葬南郡或%
曰冢在山陽高平之西也%
]%
其後裔當春秋時有任宿%
須句顓㬰皆風姓之胤也女媧氏亦風姓蛇身%
人首有神聖之德代宓犧立號曰女希氏無革%
造惟作笙簧|[%
按禮明堂位及丗%
本皆云女媧作簧%
]%
故易不載不承五%
運一曰女媧亦木德王盖宓犧之後已經數丗%
金木輪環周而復始特舉女媧以其功髙而充%
三皇故頻木王也當其末年也諸侯有共工氏%
任智刑以強霸而不王以水乗木乃與祝融戰%
不勝而怒乃頭觸不周山崩天柱折地維缺女%
媧乃鍊五色石以補天斷鼇足以立四極聚蘆%
灰以止滔水以濟兾州|[%
按其事岀%
淮南子也%
]%
於是地平天成%
不改舊物女媧氏沒神農氏作|[%
按三皇記者不同%
譙周以燧人爲皇%
%
宋均以祝融爲皇而鄭玄依春秋緯以女媧%
爲皇承伏犧皇甫謐亦同今依之爲說也%
]%
炎帝神農氏%
姜姓母曰女登有媧氏之女爲少典妃感神龍%
而生炎帝人身牛首長於姜水因以爲姓|[%
按國%
語炎%
%
帝黃帝皆少典之子其母又皆有媧氏之女據諸子及古史%
考炎帝之後凢八代五百餘年軒轅氏代之豈炎帝黃帝是%
昆弟而同母氏乎皇甫謐以爲少典有媧氏諸侯國號然則%
姜姬二帝同岀少典氏黃帝之母又是神農母氏之後代女%
所同是有媧%
氏之女也%
]%
火德王故曰炎帝以火名官斵木爲%
耜揉木爲耒耒耨之用以教萬人始教耕故號%
神農氏於是作蜡祭以赭鞭鞭草木始甞百草%
始有醫藥又作五弦之瑟教人日中爲市交易%
而退各得其所遂重八卦爲六十四爻初都陳%
後居曲阜|[%
按今淮陽有神農井又左%
傳魯有大庭氏之庫是也%
]%
立一百二十年%
崩葬長沙神農本起烈山故左氏稱烈山氏之%
子曰柱亦曰厲山氏禮曰厲山氏之有天下是%
也|[%
按鄭玄云厲山神農所起亦曰有烈%
山皇甫謐曰厲山今隨之厲鄉也%
]%
神農納奔水氏%
之女曰聽詙爲妃生帝哀哀生帝克克生帝榆%
罔凡八代五百三十年而軒轅氏興焉|[%
按神農之%
後凡八代%
%
事見帝王代紀及古史考然古典亡矣況譙皇二氏皆前聞%
君子考按古書而爲此說豈至今鑿空乎此紀亦據以爲說%
%
其易稱神農氏沒即榆罔%
榆罔猶襲神農之號也%
]%
其後有州甫甘許戲露齊%
紀怡向申呂皆姜姓之後並爲諸侯或分四岳%
當周室甫侯申伯爲王賢相齊許列爲諸侯霸%
於中國盖聖人德澤廣大故其祚胤繁昌乆長%
云一說三皇謂天皇地皇人皇爲三皇旣是開%
闢之初君臣之始圖緯所載不可全弃故兼序%
之天地初立有天皇氏十二頭澹泊無所施爲%
而俗自化木德王歳起攝提兄弟十二人立各%
一萬八千歳|[%
盖天地初立神人首岀行化故其年丗長乆%
也然言十二頭者非謂一人之身有十二頭%
%
盖古質比之鳥%
獸頭數故也%
]%
地皇十一頭火德王姓十一人興%
於熊耳龍門等山亦各萬八千歳人皇九頭乗%
雲車駕六羽岀谷口兄弟九人分長九州各立%
城邑凡一百五十丗合四萬五千六百年|[%
天皇已%
下皆岀%
%
河圖及三%
五曆也%
]%
自人皇已後有五龍氏|[%
五龍氏兄弟五人並%
乗龍上下故曰五龍%
%
氏%
也%
]%
燧人氏|[%
按其君鑽燧岀火教人熟食在伏%
犧氏前譙周以爲三皇之首也%
]%
大庭氏栢%
皇氏中央氏卷須氏栗陸氏驪連氏赫胥氏尊%
盧氏渾沌氏昊英氏有巢氏朱襄氏葛天氏隂%
康氏無懷氏斯盖三皇已來有天下者之號|[%
按%
皇%
%
甫謐以爲大庭已下一十五君皆襲庖犧之號事不經見難%
可依從然按古封太山者首有無懷氏乃在太昊之前豈得%
%
如謐%
所說%
]%
但載籍不紀莫知姓王年代所都之處而%
韓詩以爲自古封太山禪梁甫者萬有餘家仲%
尼觀之不能盡識管子亦曰古封太山七十二%
家夷吾所識十有二焉首有無懷氏然則無懷%
之前天皇已後年紀悠邈皇王何昇而告但古%
書亡矣不可備論豈得謂無帝王耶故春秋緯%
稱自開闢至於獲麟凡三百二十七萬六千歳%
分爲十紀凢丗七萬六百年一曰九頭紀二曰%
五龍紀三曰攝提紀四曰合雒紀五曰連通紀%𨿅
六曰序命紀七曰脩飛紀八曰囬提紀九曰樿%
通紀十曰流訖紀盖流訖當黄帝時制九紀之%
間是以録於此補紀之也
