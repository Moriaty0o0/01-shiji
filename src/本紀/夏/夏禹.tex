夏禹|[%
謚法曰受禪成功曰禹。正義曰夏者帝禹封國號%
也帝王紀云禹受封爲夏伯在豫州外方之南今河%
%
南陽翟%
是也%
]名曰文命|[%
索隱曰尚書云文命敷于四海孔安國%
云外布文德教命不云是禹名太史公%
%
皆以放勳重華文命爲名孔又云虞氏舜名則堯舜禹湯皆%
名矣蓋古質帝王之號皆以名後代因其行追而爲謚其實%
%
禹是名故張晏云少昊已前天下之號象其德顓頊已來天%
下之號因其名又按系本鯀取有辛氏女謂之女志是生髙%
%
密宋衷云髙密禹所封國。正義曰帝王紀云父鯀妻脩己%
見流星貫昴夢接意感又吞神珠薏苡胷坼而生禹名文命%
%
字密身九尺二寸長本西夷人也大戴禮云髙陽之孫鯀之%
子曰文命楊雄蜀王本紀云禹本汝山郡廣柔縣人也生於%汶
%
石紐括地志云茂洲汶川縣石紐山在縣西七十三里華陽%
國志云今夷人共營其地方百里不敢居牧至今猶不敢放%
%
六畜按廣柔%
隋改曰汶川%
]禹之父曰鯀鯀之父曰帝顓頊|[%
索隱%
曰皇%
%
甫謐云鯀帝顓頊之子字熙又連山易云鯀封於崇故國語%
謂之崇伯鯀系本亦以鯀爲顓頊子漢書律曆志則云顓頊%
%
五代而生鯀按鯀既仕堯與舜代系殊懸舜即顓頊%
六代孫則鯀非是顓頊之子蓋班氏之言近得其實%
]顓頊%
之父曰昌意昌意之父曰黄帝禹者黄帝之玄%
孫而帝顓頊之孫也禹之曾大父昌意及父鯀%
皆不得在帝位爲人臣當帝堯之時鴻水滔天%
|[%
索隱曰一作洪鴻大也以鳥大曰鴻小%
曰鴈故近代文字大義者皆作鴻也%
]浩浩懷山襄陵%
下民其憂堯求能治水者羣臣四嶽皆曰鯀可%
堯曰鯀爲人負命毀族不可四嶽曰等之未有%
賢於鯀者願帝試之於是堯聽四嶽用鯀治水%
九年而水不息功用不成於是帝堯乃求人更%
得舜舜登用攝行天子之政廵狩行視鯀之治%
水無狀|[%
索隱曰言%
無功狀%
]乃殛鯀於羽山以死|[%
正義曰殛音%
紀力反鯀之%
%
羽山化爲黄熋入于羽淵熋音乃來反下三%
點爲三足也束晳發蒙紀云鼈三足曰熋%
]天下皆以舜%
之誅爲是於是舜舉鯀子禹而使續鯀之業堯%
崩帝舜問四嶽曰有能成美堯之事者使居官%
皆曰伯禹爲司空可成美堯之功舜曰嗟然命%
禹女平水土維是勉之禹拜稽首讓於契后稷%
臯陶舜曰女其徃視爾事矣禹爲人敏給克勤%
其德不違其仁可親其言可信聲爲律|[%
索隱曰言%
禹聲音應%
%
鍾%
律%
]身爲度|[%
王肅曰以身爲法度。索%
隱曰按今巫猶稱禹步%
]稱以岀|[%
徐廣曰一%
作士。索%
%
隱曰按大戴禮見作士又一解云上聲與身%
爲律度則權衡亦岀於其身故云稱以岀%
]亹亹穆穆%
爲綱爲紀禹乃遂與益后稷奉帝命命諸侯百%
姓興人徒以傅土行山表木|[%
尚書傅字作敷馬融曰%
敷分也。索隱曰大戴%
%
禮作傅土故此紀依之傅即付也謂付功屬役之事謂令人%
分布理九州之土地也表木謂刊木立爲表記尚書作隨山%
%
刊%
木%
]定髙山大川|[%
馬融曰定其差秩祀禮所視也駰案尚%
書大傳曰髙山大川五嶽四瀆之屬%
]%
禹傷先人父鯀功之不成受誅乃勞身焦思居%
外十三年過家門不敢入薄衣食致孝于鬼神%
|[%
馬融曰祭%
祀豐絜%
]卑宮室致費於溝淢|[%
包氏曰方里爲井井間%閒
有溝溝廣深四尺十里%
%
爲成成間有淢%閒
淢廣深八尺%
]陸行乗車水行乗船泥行乘橇集解徐廣曰他書或作蕝。駰案孟康曰橇形如箕擿行泥上。如淳曰橇音茅蕝之蕝。謂以板置其泥上以通行路也。正義按橇形如船而短小兩頭微起人曲一腳泥上擿進用拾泥上之物。今杭州溫州海邊有之也。山行乘檋。集解徐廣曰檋一作橋音丘遙反。駰案如淳曰檋車謂以鐵如錐頭長半寸施之履下以上山不蹉跌也。又音紀錄反。正義按上山前齒短後齒長下山前齒長後齒短也。檋音與是同也。左準繩右規矩集解王肅曰左右言常用也。索隱左所運用堪爲人之準繩右所舉動必應規矩也。載四時集解王肅曰所以行不違四時之宜也。以開九州通九道陂九澤度九山。令益予衆庶稻可種卑溼。命后稷予衆庶難得之食。食少調有餘相給以均諸侯。禹乃行相地宜所有以貢及山川之便利。

禹行自冀州始。冀州既載集解孔安國曰堯所都也。先施貢賦役載於書也。鄭玄曰兩河閒曰冀州。正義按理水及貢賦從帝都爲始也。黃河自勝州東直南至華陰即東至懷州南又東北至平州碣石山入海也。東河之西西河之東南河之北皆冀州也。壺口治梁及岐。集解鄭玄曰地理志壺口山在河東北屈縣之東南梁山在左馮翊夏陽岐山在右扶風美陽。索隱鄭玄曰地理志壺口山在河東北屈縣之東南梁山在左馮翊夏陽岐山在右扶風美陽。正義括地志云壺口山在慈州吉昌縣西南五十里冀州境也。梁山在同州韓城縣東南十九里岐山在岐州岐山縣東北十里二山雍州境也孔安國曰從東循山理水而西也。既脩太原至于嶽陽。集解孔安國曰太原今爲郡名。太嶽在太原西南。山南曰陽。索隱嶽太嶽即冀州之鎭霍太山也。按地理志霍太山在河東彘縣東。凡如此例不引書者皆地理志文也。正義括地志云霍太山在沁州沁原縣西七八十里。覃懷致功集解孔安國曰覃懷近河地名。鄭玄曰懷縣屬河內。索隱按河內有懷縣今驗地無名覃者蓋覃懷二字或當時共爲一地之名。至於衡漳。集解孔安國曰漳水橫流。索隱案孔注以衡爲橫非。王肅云衡漳二水名。地理志清漳水岀上黨沾縣東北至阜城縣入河。濁漳水岀上黨長子縣東至鄴入清漳也。正義括地志云故懷城在懷州武陟縣西十一里。衡漳水在瀛州東北百二十五里平舒縣界也。其土白壤。集解孔安國曰土無塊曰壤。賦上上錯集解孔安國曰上上第一。錯雜也雜岀第二之賦。田中中集解孔安國曰九州之中爲第五。常衞既從大陸既爲。集解鄭玄曰地理志恆水岀恆山衞水在靈壽大陸澤在鉅鹿。索隱此文改恆山恆水皆作常避漢文帝諱故也。常水岀常山上曲陽縣東入滱水。衞水岀常山靈壽縣東入虖池。郭璞云大陸今鉅鹿北廣河澤是已。爲亦作也。鳥夷皮服。集解鄭玄曰鳥夷東北方之民賦搏食鳥獸者。孔安國曰服其皮明水害除。正義括地志云靺鞨國古肅慎也在京東北萬里已下東及北各抵大海。其國南有白山鳥獸草木皆白。其人處山林閒土氣極寒常爲穴居以深爲貴至接九梯。養豕食肉衣其皮冬以豬膏塗身厚數分以禦風寒。貴臭穢不絜作廁於中圜之而居。多勇力善射。弓長四尺如弩矢用楛長一尺八寸青石爲鏃。葬則交木作椁殺豬積椁上富者至數百貧者數十以爲死人之糧。以土上覆之以繩繫於椁。頭岀土上以酒灌酹繩腐而止無四時祭祀也。夾右碣石集解孔安國曰碣石海畔之山也。入于海。集解徐廣曰海一作河。索隱地理志云碣石山在北平驪城縣西南。太康地理志云樂浪遂城縣有碣石山長城所起。又水經云在遼西臨渝縣南水中。蓋碣石山有二此云夾右碣石入于海當是北平之碣石。

濟河維沇州集解鄭玄曰言沇州之界在此兩水之閒。九河既道集解馬融曰九河名徒駭太史馬頰覆釜胡蘇簡絜鉤盤鬲津。雷夏既澤雍沮會同集解鄭玄曰雍水沮水相觸而合入此澤中地理志曰雷澤在濟陰城陽縣西北。索隱爾雅云水自河岀爲雍也。正義括地志云雷夏澤在濮州雷澤縣郭外西北。雍沮二水在雷澤西北平地也。桑土既蠶於是民得下丘居土。集解孔安國曰大水去民下丘居平土就桑蠶。其土黑墳集解孔安國曰色黑而墳起。草繇木條。集解孔安國曰繇茂條長也。田中下集解孔安國曰第六。賦貞作十有三年乃同。集解鄭玄曰貞正也。治此州正作不休十三年乃有賦與八州同言功難也。其賦下下。其貢漆絲其篚織文。集解孔安國曰地宜漆林又宜桑蠶。織文錦綺之屬盛之筐篚而貢焉。浮於濟漯通於河。集解鄭玄曰地理志云漯水岀東郡東武陽。索隱濟水岀河東垣縣王屋山東其流至濟陰故應劭云濟水岀平原漯陰縣東漯水岀東郡東武陽縣北至千乘縣而入于海。

海岱維青州集解鄭玄曰東自海西至岱。東嶽曰岱山。正義按舜分青州爲營州遼西及遼東。堣夷既略集解馬融曰堣夷地名。用功少曰略。索隱孔安國云東表之地稱嵎夷。按今文尚書及帝命驗並作禺鐵在遼西。鐵古夷字也。濰淄其道。集解鄭玄曰地理志濰水岀琅邪淄水岀泰山萊蕪縣原山。索隱濰水岀琅邪箕縣北至都昌縣入海。淄水岀泰山萊蕪縣原山北東至博昌縣入濟也。正義括地志云密州莒縣濰山濰水所岀。淄州淄川縣東北七十里原山淄水所岀。俗傳云禹理水功畢土石黑數里之中波若漆故謂之淄水也。其土白墳海濱廣潟集解徐廣曰一作澤又作斥。厥田斥鹵。集解鄭玄曰斥謂地鹹鹵。索隱鹵音魯。說文云鹵鹹地。東方謂之斥西方謂之鹵。田上下賦中上。集解孔安國曰田第三賦第四。厥貢鹽絺海物維錯集解孔安國曰絺細葛。錯雜非一種。鄭玄曰海物海魚也。魚種類尤雜。岱畎絲枲鉛松怪石集解孔安國曰畎谷也。怪異好石似玉者。岱山之谷岀此五物皆貢之。萊夷爲牧集解孔安國曰萊夷地名可以牧放。索隱按左傳云萊人劫孔子孔子稱夷不亂華又云齊侯伐萊服虔以爲東萊黃縣是。今按地理志黃縣有萊山恐即此地之夷。其篚酓絲。集解孔安國曰酓桑蠶絲中琴瑟弦。索隱爾雅云檿山桑是蠶食檿之絲也。浮於汶通於濟。集解鄭玄曰地理志汶水岀泰山萊蕪縣原山西南入濟。

海岱及淮維徐州集解孔安國曰東至海北至岱南及淮。淮沂其治蒙羽其蓺。集解鄭玄曰地理志沂水岀泰山蓋縣。蒙羽二山名。孔安國曰二水已治二山可以種蓺。索隱水經云淮水岀南陽平氏縣胎簪山東北過桐柏山。沂水岀泰山蓋縣艾山南過下邳縣入泗。蒙山在泰山蒙陰縣西南。羽山在東海祝其縣南殛鯀之地。大野既都集解鄭玄曰大野在山陽鉅野北名鉅野澤。孔安國曰水所停曰都。東原厎平。集解鄭玄曰東原地名。今東平郡即東原。索隱張華博物志云兗州東平郡即尚書之東原也。正義廣平曰原。徐州在東故曰東原。水去已致平復言可耕種也。其土赤埴墳集解徐廣曰埴黏土也。草木漸包。集解孔安國曰漸長進包叢生也。其田上中賦中中。集解孔安國曰田第二賦第五。貢維土五色集解鄭玄曰土五色者所以爲大社之封。正義韓詩外傳云天子社廣五丈東方青南方赤西方白北方黑上冒以黃土。將封諸侯各取方土苴以白茅以爲社也。太康地記云城陽姑幕有五色土封諸侯錫之茅土用爲社。此土即禹貢徐州土也。今屬密州莒縣也。羽畎夏狄集解孔安國曰夏狄狄雉名也。羽中旌旄羽山之谷有之。嶧陽孤桐集解孔安國曰嶧山之陽特生桐中琴瑟。鄭玄曰地理志嶧山在下邳。正義括地志云嶧山在兗州鄒縣南二十二里。鄒山記云鄒山古之嶧山言絡繹相連屬也。今猶多桐樹。按今獨生桐尚徵一偏似琴瑟。泗濱浮磬集解孔安國曰泗水涯水中見石可以爲磬。鄭玄曰泗水岀濟陰乘氏也。正義括地志云泗水至彭城呂梁岀石磬。淮夷蠙珠臮魚集解孔安國曰淮夷二水岀蠙珠及美魚。鄭玄曰淮夷淮水之上夷民也。索隱按尚書云徂茲淮夷徐戎並興今徐州言淮夷則鄭解爲得。蠙一作玭並步玄反。臮古曁字。臮與也。言夷人所居淮水之處有此蠙珠與魚也。又作濱。濱畔也。其篚玄纖縞。集解鄭玄曰纖細也。祭服之材尚細。正義玄黑。纖細。縞白繒。以細繒染爲黑色。浮于淮泗正義括地志云泗水源在兗州泗水縣東陪尾山。其源有四道因以爲名。通于河。

淮海維揚州集解孔安國曰北據淮南距海。彭蠡既都陽鳥所居。集解鄭玄曰地理志彭蠡澤在豫章彭澤西。孔安國曰隨陽之鳥鴻鴈之屬冬月居此澤也。索隱都古文尚書作豬。孔安國云水所停曰豬鄭玄云南方謂都爲豬則是水聚會之義。正義蠡音禮。括地志云彭蠡湖在江州潯陽縣東南五十二里。三江既入索隱韋昭云三江謂松江錢唐江浦陽江。今按地理志有南江中江北江是爲三江。其南江從會稽吳縣南東入海。中江從丹陽蕪湖縣西南東至會稽陽羨縣入海。北江從會稽毗陵縣北東入海。故下文東爲中江又東爲北江孔安國云有北有中南可知也。震澤致定。集解孔安國曰震澤吳南太湖名。言三江已入致定爲震澤。索隱震一作振。地理志會稽吳縣故周泰伯所封國具區在其西古文以爲震澤。又左傳稱笠澤亦謂此也。正義澤在蘇州西南四十五里。三江者在蘇州東南三十里名三江口。一江西南上七十里至太湖名曰松江古笠澤江一江東南上七十里至白蜆湖名曰上江亦曰東江一江東北下三百餘里入海名曰下江亦曰婁江於其分處號曰三江口。顧夷吳地記云松江東北行七十里得三江口。東北入海爲婁江東南入海爲東江幷松江爲三江是也。言理三江入海非入震澤也。按太湖西南湖州諸溪從天目山下西北宣州諸山有溪並下太湖。太湖東北流各至三江口入海。其湖無通彭蠡湖及太湖處並阻山陸。諸儒及地志等解三江既入皆非也。周禮職方氏云揚州藪曰具區川曰三江。按五湖三江者韋昭注非也。其源俱不通太湖引解三江既入失之遠矣。五湖者菱湖游湖莫湖貢湖胥湖皆太湖東岸五灣爲五湖蓋古時應別今並相連。菱湖在莫釐山東周迴三十餘里西口闊二里其口南則莫釐山北則徐侯山西與莫湖連。莫湖在莫釐山西及北北與胥湖連胥湖在胥山西南與莫湖連各周迴五六十里西連太湖。游湖在北二十里在長山東湖西口闊二里其口東南岸樹里山西北岸長山湖周迴五六十里。貢湖在長山西其口闊四五里口東南長山山南即山陽村西北連常州無錫縣老岸湖周迴一百九十里已上湖身向東北長七十餘里。兩湖西亦連太湖。河渠書云於吳則通渠三江五湖。貨殖傳云夫吳有三江五湖之利。又太史公自敍傳云登姑蘇望五湖是也。竹箭既布。集解孔安國曰水去布生。其草惟夭其木惟喬集解少長曰夭。喬髙也。其土塗泥。集解馬融曰漸洳也。田下下賦下上上雜。集解孔安國曰田第九賦第七雜岀第六。貢金三品集解孔安國曰金銀銅。。鄭玄曰銅三色也。瑤琨竹箭集解孔安國曰瑤琨皆美玉也。齒革羽旄集解孔安國曰象齒犀皮鳥羽旄牛尾也。正義周禮考工記云犀甲七屬兕甲六屬。郭云犀似水牛豬頭大腹庳腳橢角好食𣗥也。亦有一角者。按西南夷常貢旄牛尾爲旌旗之飾書詩通謂之旄。故尚書云右秉白旄詩云建旐設旄皆此牛也。島夷卉服集解孔安國曰南海島夷草服葛越。正義括地志云百濟國西南渤海中有大島十五所皆邑落有人居屬百濟。又倭國武皇后改曰日本國在百濟南隔海依島而居凡百餘小國。此皆揚州之東島夷也。按東南之夷草服葛越焦竹之屬越即苧祁也。其篚織貝集解孔安國曰織細繒也。貝水物也。鄭玄曰貝錦名也。詩云成是貝錦。凡織者先染其絲織之屬百濟。又倭國武皇后改即成文矣。其包橘柚錫貢。集解孔安國曰小曰橘大曰柚。錫命乃貢言不常也。鄭玄曰有錫則貢之或時乏則不貢。錫所以柔金也。均江海通淮泗。集解鄭玄曰均讀曰沿。沿順水行也。

荊及衡陽維荊州集解孔安國曰北據荊山南及衡山之陽。江漢朝宗于海。集解孔安國曰二水經此州而入海有似於朝百川以海爲宗。宗尊也。正義括地志云江水源岀岷州南岷山南流至益州即東南流入蜀至瀘州東流經三硤過荊州與漢水合。孫卿子云江水其源可以濫觴也。又云漢水源岀梁州金牛縣東二十八里嶓冢山。九江甚中集解孔安國曰江於此州界分爲九道甚得地勢之中。鄭玄曰地理志九江在尋陽南皆東合爲大江。索隱按尋陽記九江者烏江蚌江烏白江嘉靡江沙江畎江廩江隄江箘江。又張湞九江圖所載有三里五畎烏土白蚌。九江之名不同。沱涔已道集解孔安國曰沱江別名。涔水名。鄭玄曰水岀江爲沱漢爲涔。索隱涔亦作潛。沱岀蜀郡郫縣西東入江。潛岀漢中安陽縣直西南北入漢。故爾雅云水自江岀爲沱漢岀爲潛。正義括地志云繁江水受郫江。禹貢曰岷山導江東別爲沱源岀益州新繁縣。潛水一名復水今名龍門水源岀利州緜谷縣東龍門山大石穴下也。雲土夢爲治。集解孔安國曰雲夢之澤在江南其中有平土丘水去可爲耕作畎畝之治。索隱夢一作瞢鄒誕生又音蒙。按雲土夢本二澤名蓋人以二澤相近或合稱雲夢耳。知者據左傳云楚子濟江入于雲中又楚子鄭伯田于江南之夢則是二澤各別也。韋昭曰雲土今爲縣屬江夏南郡華容。今按地理志云江夏有雲杜縣是其地。其土塗泥。田下中賦上下。集解孔安國曰田第八賦第三。貢羽旄齒革金三品杶榦栝柏集解鄭玄曰四木名。孔安國曰榦柘也。柏葉松身曰栝。礪砥砮丹集解孔安國曰砥細於礪皆磨石也。砮石中矢鏃。丹朱類也。維箘簬楛集解徐廣曰一作箭足杆。杆即楛也音怙。箭足者矢鏃也。或以箭足訓釋箘簬乎駰案鄭玄曰箘簬聆風也。三國致貢其名集解馬融曰言箘簬楛三國所致貢其名善也。包匭菁茅集解鄭玄曰匭纏結也。菁茅茅有毛刺者給宗廟縮酒。重之故包裹又纏結也。正義括地志云辰州盧溪縣西南三百五十里有包茅山。武陽記云山際岀包茅有刺而三脊因名包茅山。其篚玄纁璣組集解孔安國曰此州染玄纁色善故貢之。璣珠類生於水中。組綬類也。九江入賜大龜。集解孔安國曰尺二寸曰大龜岀於九江水中。龜不常用賜命而納之。浮于江沱涔于漢踰于雒至于南河。

荊河惟豫州集解孔安國曰西南至荊山北距河水。正義括地志云荊山在襄州荊山縣西八十里。韓子云卞和得玉璞於楚之荊山即此也。河洛州北河也。伊雒瀍澗既入于河集解孔安國曰伊岀陸渾山洛岀上洛山澗岀澠池山瀍岀河南北山四水合流而入河。索隱伊水岀弘農盧氏縣東洛水岀弘農上洛縣冢領山瀍水岀河南穀城縣朁亭北澗水岀弘農新安縣東皆入于河。正義括地志云伊水岀虢州盧氏縣東巒山東北流入洛。洛水岀商州洛南縣冢領山東流經洛州郭內又東合伊水。瀍水岀洛州新安縣東南流至洛州郭內南入洛。澗水源岀洛州新安縣東白石山東北與穀水合流經洛州郭內東流入洛也。滎播既都集解孔安國曰滎澤名。波水已成遏都。索隱古文尚書作滎波此及今文並云滎播。播是水播溢之義滎是澤名。故左傳云狄及衞戰於滎澤。鄭玄云今塞爲平地滎陽人猶謂其處爲滎播。道荷澤被明都。集解孔安國曰荷澤在胡陵。明都澤名在河東北水流泆覆被之。索隱荷澤在濟陰定陶縣東。明都音孟豬。孟豬澤在梁國睢陽縣東北。爾雅左傳謂之孟諸今文亦爲然唯周禮稱望諸皆此地之一名。正義括地志云荷澤在曹州濟陰縣東北九十里定陶城東今名龍池亦名九卿陂。其土壤下土墳壚。集解孔安國曰壚疏也。馬融曰豫州地有三等下者墳壚也。田中上賦雜上中。集解孔安國曰田第四賦第二又雜岀第一。貢漆絲絺紵其篚纖絮集解孔安國曰細緜也。錫貢磬錯。集解孔安國曰治玉石曰錯治磬錯也。浮於雒達於河。

華陽黑水惟梁州集解孔安國曰東據華山之南西距黑水。正義括地志云黑水源岀梁州城固縣西北太山。汶嶓既蓺集解鄭玄曰地理志岷山在蜀郡湔氐道嶓冢山在漢陽西。索隱汶一作崏又作㞶。㞶山封禪書一云瀆山在蜀都湔氐道西徼江水所岀。嶓冢山在隴西西縣漢水所岀也。正義括地志云岷山在岷州溢樂縣南一里連緜至蜀二千里皆名岷山。嶓冢山在梁州金牛縣東二十八里。湔音子踐反。氐音丁奚反。沱涔既道集解孔安國曰沱潛發源此州入荊州。蔡蒙旅平集解孔安國曰蔡蒙二山名。祭山曰旅。平言治功畢也。鄭玄曰地理志蔡蒙在漢嘉縣。索隱此非徐州之蒙在蜀郡青衣縣。青衣後改爲漢嘉。蔡山不知所在也。蒙縣名。正義括地志云蒙山在雅州嚴道縣南十里。和夷厎績。集解馬融曰和夷地名也。其土青驪。集解孔安國曰色青黑也。田下上賦下中三錯。集解孔安國曰田第七賦第八雜岀第七第九三等。貢璆鐵銀鏤砮磬集解孔安國曰璆玉名。鄭玄曰黃金之美者謂之鏐。鏤剛鐵可以刻鏤也。熊羆狐貍織皮。集解孔安國曰貢四獸之皮也。織皮今罽也。西傾因桓是來集解馬融曰治西傾山因桓水是來言無餘道也。鄭玄曰地理志西傾山在隴西臨洮。索隱西傾在隴西臨洮縣西南。桓水岀蜀郡㞶山西南行羌中入南海也。正義括地志云西傾山今嵹臺山在洮州臨潭縣西南三百三十六里。浮于潛踰于沔集解孔安國曰漢上水爲沔。鄭玄曰或謂漢爲沔。入于渭亂于河。集解孔安國曰正絕流曰亂。

黑水西河惟雍州集解孔安國曰西距黑水東據河。龍門之河在冀州西。索隱地理志益州滇池有黑水祠。鄭玄引地說云三危山黑水岀其南。山海經黑水岀崑崙墟西北隅也。弱水既西集解孔安國曰導之西流至于合黎。鄭玄曰衆水皆東此獨西流也。索隱按水經云弱水岀張掖刪丹縣西北至酒泉會水縣入合黎山腹。山海經云弱水岀崑崙墟西南隅也。涇屬渭汭。集解孔安國曰屬逮也。水北曰汭。言治涇水入於渭也。鄭玄曰地理志涇水岀安定涇陽。索隱渭水岀首陽縣鳥鼠同穴山。說文云水相入曰汭。正義括地志云涇水源岀原州百泉縣西南笄頭山涇谷。渭水源岀渭州渭原縣西七十六里鳥鼠山今名青雀山。渭有三源並岀鳥鼠山東流入河。按言理涇水及至渭水又理漆沮亦從渭流復理灃水亦同入渭者也。漆沮既從正義括地志云漆水源岀岐州普潤縣東南岐漆山漆溪東入渭。沮水一名石川水源岀雍州富平縣東入櫟陽縣南。漢髙帝於櫟陽置萬年縣。十三州地理志云萬年縣南有涇渭北有小河即沮水也。詩云古公去邠度漆沮即此二水。灃集解音豐。水所同。集解孔安國曰漆沮之水已從入渭。灃水所同同于渭也。索隱漆沮二水漆水岀右扶風漆縣西沮水地理志無文而水經以𣻐水岀北地直路縣東過馮翊祋祤縣入洛。說文亦以漆沮各是一水名。孔安國獨以爲一又云是洛水。灃水岀右扶風鄠縣東南北過上林苑。正義括地志云雍州鄠縣終南山灃水岀焉。荊岐已旅集解孔安國曰荊在岐東非荊州之荊也。正義括地志云荊山在雍州富平縣今名掘陵原。岐山在岐州岐山縣東北十里。尚書正義云洪水時祭祀禮廢。已旅祭言理水功畢也。按雍州荊山即黃帝及禹鑄鼎地也。襄州荊山縣西荊山即卞和得玉璞者。終南敦物至于鳥鼠。集解孔安國曰三山名言相望也。鄭玄曰地理志終南敦物皆在右扶風武功也。索隱按左傳中南山杜預以爲終南山。地理志云太一山古文以爲終南華垂山古文以爲敦物皆在扶風武功縣東。正義括地志云終南山一名中南山一名太一山一名南山一名橘山一名楚山一名泰秦山一名周南山一名地肺山在雍州萬年縣南五十里。原隰厎績至于都野。集解鄭玄曰地理志都野在武威名曰休屠澤。正義原隰幽州地也。按原髙平地也。隰低下地也。言從渭州致功西北至涼州都野沙州三危山也。括地志云都野澤在涼州姑臧縣東北二百八十里。三危既度索隱鄭玄引河圖及地說云三危山在鳥鼠西南與岐山相連。度劉伯莊音田各反尚書作宅。三苗大序。集解孔安國曰西裔之山己可居三苗之族大有次序禹之功也。其土黃壤。田上上賦中下。集解孔安國曰田第一賦第六人功少。貢璆琳琅玕。集解孔安國曰璆琳皆玉名。琅玕石而似珠者。浮于積石至于龍門西河集解孔安國曰積石山在金城西南河所經也。龍門山在河東之西界。索隱積石在金城河關縣西南。龍門山在左馮翊夏陽縣西北。正義括地志云積石山今名小積石在河州枹罕縣西七里。河州在京西一千四百七十二里。龍門山在同州韓城縣北五十里。李奇云禹鑿通河水處廣八十步。三秦記云龍門水懸船而行兩旁有山水陸不通龜魚集龍門下數千不得上上則爲龍故云暴鰓點額龍門下。按河在冀州西故云西河也。禹發源河水小積石山浮河東北下歷靈勝北而南行至于龍門皆雍州地也。會于渭汭。正義水經云河水又南至潼關渭水從西注之也。織皮昆侖析支渠搜西戎即序。集解孔安國曰織皮毛布。此四國在荒服之外流沙之內。羌髳之屬皆就次序美禹之功及戎狄也。索隱鄭玄以爲衣皮之人居昆侖析支渠搜三山皆在西戎。王肅曰昆侖在臨羌西析支在河關西西戎在西域。王肅以爲地名而不言渠搜。今按地理志金城臨羌縣有昆侖祠敦煌廣至縣有昆侖障朔方有渠搜縣。

道九山索隱汧壺口砥柱太行西傾熊耳嶓冢內方㞶是九山也。古分爲三條故地理志有北條之荊山。馬融以汧爲北條西傾爲中條嶓冢爲南條。鄭玄分四列汧爲陰列西傾次陰列嶓冢爲陽列㞶山次陽列。汧及岐至于荊山集解鄭玄曰地理志汧在右扶風也。索隱汧一作岍。按有汧水故其字或從山或從水猶㞶山然也。地理志云吳山在汧縣西古文以爲汧山。岐山在右扶風美陽縣西北荊山在左馮翊懷德縣南也。正義括地志云汧山在隴州汧源縣西六十里。其山東鄰岐岫西接隴岡汧水岀焉。岐山在岐州。踰于河壺口雷首索隱雷首山在河東蒲阪縣東南。至于太嶽集解孔安國曰三山在冀州太嶽在上黨西也。索隱即霍泰山也。已見上。正義括地志云壺口在慈州吉昌縣西南。雷首山在蒲州河東縣。太嶽霍山也在沁州沁源縣。砥柱析城至于王屋集解孔安國曰此三山在冀州之南河之北。索隱析城山在河東濩澤縣西南。王屋山在河東垣縣東北。水經云砥柱山在河東大陽縣南河水中也。正義括地志云厎柱山俗名三門山在陝州硤石縣東北五十里黃河之中。孔安國云厎柱山名。河水分流包山而過山見水中若柱然也。括地志云析城山在澤州陽城縣西南七十里。注水經云析城山甚髙峻上平坦有二泉東濁西清左右不生草木。括地志云王屋山在懷州王屋縣北十里。古今地名云山方七百里山髙萬仞本冀州之河陽山也。太行常山至于碣石入于海集解孔安國曰此二山連延東北接碣石而入于滄海。索隱太行山在河內山陽縣西北。常山恆山是也在常山郡上曲陽縣西北。正義括地志云太行山在懷州河內縣北二十五里有羊腸阪。恆山在定州恆陽縣西北百四十里。道書福地記云恆山髙三千三百丈上方二十里有太玄之泉神草十九種可度俗。西傾朱圉鳥鼠集解鄭玄曰地理志曰朱圉在漢陽南。孔安國曰鳥鼠山渭水所山在隴西之西。至于太華集解鄭玄曰地理志太華山在弘農華陰南。索隱圉一作圄。朱圉山在天水冀縣南。鳥鼠山在隴西首陽縣西南。太華即敦物山。熊耳外方桐柏至于負尾集解鄭玄曰地理志熊耳在盧氏東。外方在潁川。嵩髙山桐柏山在南陽平氏東南。陪尾在江夏安陸東北若橫尾者。索隱熊耳山在弘農盧氏縣東伊水所岀。外方山即潁川嵩髙縣嵩髙山古文尚書亦以爲外方山。桐柏山一名大復山在南陽平氏縣東南。陪尾山在江夏安陸縣東北地理志謂之橫尾山。負音陪也。正義括地志云華山在華州華陰縣南八里。熊耳山在虢州盧氏縣南五十里。嵩髙山亦名太室山亦名外方山在洛州陽城縣北二十三里也。桐柏山在唐州桐柏縣東南五十里淮水岀焉。橫尾山古陪尾山也在安州安陸縣北六十里。道嶓冢至于荊山集解鄭玄曰地理志荊山在南郡臨沮。索隱此東條荊山在南郡臨沮縣東北隅也。正義括地志云嶓冢山在梁州。荊山在襄州荊山縣西八十里也。又云荊山縣本漢臨沮縣地也。沮水即漢水也。按孫叔敖激沮水爲雲夢澤是也。內方至于大別集解鄭玄曰地理志內方在竟陵名立章山。大別在廬江安豐縣。索隱內方山在竟陵縣東北。大別山在六安國安豐縣今土人謂之甑山。正義括地志云章山在荊州長林縣東北六十里。今漢水附章山之東與經史符會。按大別山今沙洲在山上漢江經其左今俗猶云甑山。注云在安豐非漢所經也。汶山之陽至衡山索隱在長沙湘南縣東南。廣雅云岣嶁謂之衡山。正義括地志云岷山在茂州汶川縣。衡山在衡州湘潭縣西四十一里。過九江至于敷淺原。集解徐廣曰淺一作滅。駰案孔安國曰敷淺原一名傅陽山在豫章。索隱豫章歷陵縣南有傅陽山一名敷淺原也。

道九川索隱弱黑河瀁江沇淮渭洛爲九川。弱水至於合黎集解鄭玄曰地理志弱水岀張掖。孔安國曰合黎水名在流沙東。索隱水經云合黎山在酒泉會水縣東北。鄭玄引地說亦以爲然。孔安國云水名當是其山有水故所記各不同。正義括地志云蘭門山一名合黎一名窮石山在甘州刪丹縣西南七十里。淮南子云弱水源岀窮石山。又云合黎一名羌谷水一名鮮水一名覆表水今名副投河亦名張掖河南自吐谷渾界流入甘州張掖縣。今按合黎水岀臨松縣臨松山東而北流歷張掖故城下又北流經張掖縣二十三里又北流經合黎山折而北流經流沙磧之西入居延海行千五百里。合黎山張掖縣西北二百里也。餘波入于流沙。集解孔安國曰弱水餘波西溢入流沙。鄭玄曰地理志流沙在居延西東北名居延澤。地記曰弱水西流入合黎山腹餘波入于流沙通于南海。馬融王肅皆云合黎流沙是地名。索隱地理志云張掖居延縣西北有居延澤古文以爲流沙。廣志流沙在玉門關外有居延澤居延城。又山海經云流沙岀鐘山西南行昆侖墟入海。按是地兼有水故一云地名一云水名馬鄭不同抑有由也。道黑水至于三危入于南海。集解鄭玄曰地理志益州滇池有黑水祠而不記此山水所在。地記曰三危山在鳥鼠之西南。孔安國曰黑水自北而南經三危過梁州入南海也。正義括地志云黑水源岀伊州伊吾縣北百二十里又南流二千里而絕。三危山在沙州燉煌縣東南四十里。按南海即揚州東大海岷江下至揚州東入海也。其黑水源在伊州從伊州東南三千餘里至鄯州鄯州東南四百餘里至河州入黃河。河州有小積石山即禹貢浮於積石至於龍門者。然黃河源從西南下岀大崑崙東北隅東北流經于闐入鹽澤即東南潛行入吐谷渾界大積石山又東北流至小積石山又東北流來處極遠。其黑水當洪水時合從黃河而行何得入于南海南海去此甚遠阻隔南山隴山岷山之屬。當是洪水浩浩處西戎不深致功古文故有疏略也。道河積石索隱爾雅云河岀昆侖墟其色白。漢書西域傳云河有兩源一岀蔥嶺一岀于闐。于闐河北流與蔥嶺河合東注蒲昌海一名鹽澤。其水停居冬夏不增減潛行地中南岀積石爲中國河。是河源發昆侖禹導河自積石而加功也。至于龍門南至華陰集解孔安國曰至華山北而東行。正義華陰縣在華山北本魏之陰晉縣秦惠文王更名寧秦漢髙帝改曰華陰。東至砥柱集解孔安國曰砥柱山名。河水分流包山而過山見水中若柱然也。在西虢之界。正義砥柱山俗名三門山禹鑿此山三道河水故曰三門也。又東至于盟津集解孔安國曰在洛北。索隱盟古孟字。孟津在河陽。十三州記云河陽縣在河上即孟津是也。正義杜預云盟河內郡河陽縣南孟津也在洛陽城北。都道所湊古今爲津武王度之近代呼爲武濟。括地志云盟津周武王伐紂與八百諸侯會盟津。亦曰孟津又曰富平津。水經云小平津今云河陽津是也。東過雒汭至于大邳集解孔安國曰洛汭洛入河處。山再成曰邳。索隱爾雅云山一成曰邳。或以爲成皋縣山是。正義李廵云山再重曰英一重曰邳。括地志云大邳山今名黎陽東山又曰青檀山在衞州黎陽南七里。張揖云今成皋非也。北過降水至于大陸集解鄭玄曰地理志降水在信都南。孔安國曰大陸澤名。索隱地理志降水字從系岀信都國與虖池漳河水並流入海。大陸在鉅鹿郡。爾雅云晉有大陸郭璞以爲此澤也。正義括地志云降水源岀潞州屯留縣西南東北流至冀州入海。北播爲九河同爲逆河集解鄭玄曰下尾合名曰逆河言相向迎受也。入于海。正義播布也。河至冀州分布爲九河下至滄州更同合爲一大河名曰逆河而夾右碣石入于渤海也。嶓冢道瀁東流爲漢集解鄭玄曰地理志瀁水岀隴西氐道至武都爲漢至江夏謂之夏水。索隱水經云瀁水岀隴西氐道縣嶓冢山東至武都沮縣爲漢水。地理志云至江夏謂之夏水。山海經亦以漢岀嶓冢山。故孔安國云泉始岀山爲瀁水東南流爲沔水至漢中東流爲漢水。正義括地志云嶓冢山水始岀山沮洳故曰沮水。東南爲瀁水又爲沔水。至漢中爲漢水至均州爲滄浪水。始欲岀大江爲夏口又爲沔口。漢江一名沔江也。又東爲蒼浪之水集解孔安國曰別流也。在荊州。索隱馬融鄭玄皆以滄浪爲夏水即漢河之別流也。漁父歌曰滄浪之水清兮可以濯吾纓是此水也。正義括地志云均州武當縣有滄浪水。庾仲雍漢水記云武當縣西四十里漢水中有洲名滄浪洲也。地記云水岀荊山東南流爲滄浪水。過三澨入于大別集解孔安國曰三澨水名。鄭玄曰在江夏竟陵之界。索隱水經云三澨地名在南郡邔縣北。孔安國鄭玄以爲水名。今竟陵有三參水俗云是三澨水。參音去聲。南入于江東匯澤爲彭蠡集解孔安國曰匯回也。水東回爲彭蠡大澤。東爲北江入于海。集解孔安國曰自彭蠡江分爲三道入震澤遂爲北江而入海。汶山道江東別爲沱又東至于醴集解孔安國及馬融王肅皆以醴爲水名。鄭玄曰醴陵名也。大阜曰陵。長沙有醴陵縣。索隱按騷人所歌濯余佩於醴浦明醴是水。孔安國馬融解得其實。又虞喜志林以醴是江沅之別流而醴字作澧也。過九江至于東陵集解孔安國曰東陵地名。東迆北會于匯集解孔安國曰迆溢也。東溢分流都共北會彭蠡。東爲中江入于海。集解孔安國曰有北有中南可知也。正義括地志云禹貢三江俱會于彭蠡合爲一江入于海。道沇水東爲濟入于河泆爲滎集解鄭玄曰地理志沇水岀河東垣縣東王屋山東至河內武德入河泆爲滎。孔安國曰濟在溫西北。滎澤在敖倉東南。索隱水經云自河東垣縣王屋山東流爲沇水至溫縣西北爲濟水。正義括地志云沇水岀懷州王屋縣北十里王屋山頂巖下石泉渟不流其深不測既見而伏至濟源縣西北二里平地其源重發而東南流爲汜水。水經云沇東至溫縣西北爲泲水又南當鞏縣之北南入于河。釋名云濟者濟也。下濟子細反。按濟水入河而南截度河南岸溢滎澤在鄭州滎澤縣西北四里。今無水成平地。東岀陶丘北集解孔安國曰陶丘丘再成者也。鄭玄曰地理志陶丘在濟陰定陶西北南。正義括地志云陶丘在濮州鄄城西南二十四里。又云在曹州城中。徐才宗國都城記云此城中髙丘即古之陶丘。又東至于荷集解孔安國曰荷澤之水。又東北會于汶正義汶音問。地埋志云汶水岀泰山郡萊蕪縣原山西南入泲。又東北入于海。道淮自桐柏正義地埋志云桐柏山在南陽平氏縣東南淮水所岀。按在唐州東五十餘里。東會于泗沂東入于海。集解孔安國曰與泗沂二水合入海。道渭自鳥鼠同穴集解孔安國曰鳥鼠共爲雄雌同穴處此山遂名曰鳥鼠渭水岀焉。正義括地志云鳥鼠山今名青雀山在渭州渭源縣西七十六里。山海經云鳥鼠同穴之山渭水岀焉。郭璞注云今在隴西首陽縣西南。山有鳥鼠同穴。鳥名䳜。鼠名鼣如人家鼠而短尾。䳜似鵽而小黃黑色。穴入地三四尺鼠在內鳥在外。䳜音余。鼣扶廢反。鵽音丁刮反似雉也。東會于灃正義灃音豐。括地志云雍州鄠縣終南山灃水岀焉北入渭也。又東北至于涇正義括地志云涇水岀原州百泉縣西南笄頭山岀涇谷東南流入渭也。東過漆沮入于河。集解孔安國曰漆沮二水名亦曰洛水岀馮翊北。道雒自熊耳集解孔安國曰在宜陽之西。正義括地志云洛水岀商州洛南縣西冢嶺山東北流入河。熊耳山在虢州盧氏縣南五十里洛所經。東北會于澗瀍集解孔安國曰會于河南城南。正義括地志云澗水岀洛州新安縣東白石山陰。地理志云瀍水岀河南穀城縣朁亭北東南入於洛。又東會于伊集解孔安國曰會於洛陽之南。東北入于河。集解孔安國曰合於鞏之東也。


於是九州攸同四奧既居集解孔安國曰四方之宅已可居也。九山栞旅集解孔安國曰九州名山已槎木通道而旅祭也。九川滌原集解孔安國曰九州之川已滌除無壅塞也。九澤既陂集解孔安國曰九州之澤皆已陂障無決溢也。四海會同。六府甚脩集解孔安國曰六府金木水火土穀。衆土交正致慎財賦集解鄭玄曰三壤上中下各三等也。咸則三壤成賦。集解鄭玄曰衆土美惡及髙下得其正矣。亦致其貢篚慎奉其財物之稅皆法定制而入之也。中國賜土姓祗台德先不距朕行。集解鄭玄曰中即九州也。天子建其國諸侯祚之土賜之姓命之氏其敬悅天子之德既先又不距違我天子政教所行。

令天子之國以外五百里甸服集解孔安國曰爲天子之服治田去王城面五百里內。百里賦納緫集解孔安國曰甸內近王城者。禾槀曰緫供飼國馬也。索隱說文云緫聚束草也。二百里納銍集解孔安國曰所銍刈謂禾穗。索隱說文云銍穫禾短鎌也。三百里納秸服集解孔安國曰秸槀也。服槀役。索隱禮郊特牲云蒲越槀秸之美則秸是槀之類也。四百里粟五百里米。集解孔安國曰所納精者少麤者多。甸服外五百里侯服集解孔安國曰侯候也。斥候而服事也。百里采集解馬融曰采事也。各受王事者。二百里任國集解孔安國曰任王事者。三百里諸侯。集解孔安國曰三百里同爲王者斥候故合三爲一名。侯服外五百里綏服集解孔安國曰綏安也。服王者政教。三百里揆文教集解孔安國曰揆度也。度王者文教而行之三百里皆同。二百里奮武衞。集解孔安國曰文教之外二百里奮武衞天子所以安。綏服外五百里要服集解孔安國曰要束以文教也。三百里夷集解孔安國曰守平常之教事王者而已。二百里蔡。集解馬融曰蔡法。受王者刑法而已。要服外五百里荒服集解馬融曰政教荒忽因其故俗而治之。三百里蠻集解馬融曰蠻慢也。禮簡怠慢來不距去不禁。二百里流。集解馬融曰流行無城郭常居。

東漸于海西被于流沙朔南曁集解鄭玄曰朔北方也。聲教訖于四海。於是帝錫禹玄圭以告成功于天下。正義帝堯也。玄水色。以禹理水功成故錫玄圭以表顯之。自此已上並尚書禹貢文。天下於是太平治。

皋陶作士以理民。正義士若大理卿也。帝舜朝禹伯夷皋陶相與語帝前。皋陶述其謀曰信其道德謀明輔和。禹曰然如何皋陶曰於正義於音烏歎美之辭。慎其身脩正義絕句。思長集解孔安國曰慎脩其身思爲長久之道。敦序九族衆明髙翼近可遠在已。集解鄭玄曰次序九族而親之以衆賢明作羽翼之臣此政由近可以及遠也。禹拜美言曰然。皋陶曰於在知人在安民。禹曰吁皆若是惟帝其難之。集解孔安國曰言帝堯亦以爲難。知人則智能官人能安民則惠黎民懷之。能知能惠何憂乎驩兜何遷乎有苗何畏乎巧言善色佞人集解鄭玄曰禹爲父隱故言不及鯀。皋陶曰然於亦行有九德亦言其有德。乃言曰始事事集解孔安國曰言其人有德必言其所行事因事以爲驗。寬而栗集解孔安國曰性寬弘而能莊栗。柔而立集解孔安國曰和柔而能立事。愿而共集解孔安國曰慤愿而恭敬。治而敬擾而毅集解徐廣曰擾一作柔。駰案孔安國曰擾順也。致果爲毅。直而溫簡而廉剛而實彊而義章其有常吉哉。集解孔安國曰章明也。吉善也。日宣三德蚤夜翊明有家。集解孔安國曰三德九德之中有其三也。卿大夫稱家明行之可以爲卿大夫。日嚴振敬六德亮采有國。集解孔安國曰嚴敬也。行六德以信治政事可爲諸侯也。馬融曰亮信采事也。翕受普施九德咸事俊乂在官集解孔安國曰翕合也。能合受三六之德而用之以布施政教使九德之人皆用事。謂天子也如此則俊德理能之士並皆在官也百吏肅謹。毋教邪淫奇謀。非其人居其官是謂亂天事。索隱此取尚書皋陶謨爲文斷絕殊無次序即班固所謂疎略抵捂是也今亦不能深考。天討有辠五刑五用哉。集解孔安國曰言用五刑必當。吾言厎可行乎禹曰女言致可績行。皋陶曰余未有知思贊道哉。正義皋陶云我未有所知思之審贊於古道耳。謙辭也。已上並尚書皋陶謨文略其經不全備也。

帝舜謂禹曰女亦昌言。禹拜曰於予何言予思日孳孳。皋陶難禹曰何謂孳孳禹曰鴻水滔天浩浩懷山襄陵下民皆服於水。予陸行乘車水行乘舟泥行乘橇山行乘檋行山栞木。正義行寒孟反。栞口寒反。與益予衆庶稻鮮食。集解孔安國曰鳥獸新殺曰鮮。索隱予音與。上與謂同與之與下予謂施予之予。此禹言其與益施予衆庶之稻糧。以決九川致四海浚畎澮集解鄭玄曰畎澮田閒溝也。致之川。與稷予衆庶難得之食。食少調有餘補不足徙居。衆民乃定萬國爲治。皋陶曰然此而美也。

禹曰於帝慎乃在位安爾止。集解鄭玄曰安汝之所止無妄動動則擾民。輔德天下大應。清意以昭待上帝命天其重命用休。集解鄭玄曰天將重命汝以美應謂符瑞也。帝曰吁臣哉臣哉臣作朕股肱耳目。予欲左右有民女輔之。集解馬融曰我欲左右助民汝當翼成我也。余欲觀古人之象。日月星辰作文繡服色女明之。予欲聞六律五聲八音來始滑以岀入五言女聽。集解尚書滑字作曶音忽。鄭玄曰曶者。臣見君所秉。書思對命者也。君亦有焉以岀內政教於五官。索隱古文尚書作在治忽今文作采政忽先儒各隨字解之。今此云來始滑於義無所通。蓋來采字相近滑忽聲相亂始又與治相似因誤爲來始滑今依今文音采政忽三字。劉伯莊云聽諸侯能爲政及怠忽者是也。五言謂仁義禮智信五德之言鄭玄以爲岀納政教五官非也。予即辟女匡拂予。女無面諛。退而謗予。敬四輔臣。集解尚書大傳曰古者天子必有四鄰前曰疑後曰丞左曰輔右曰弼。諸衆讒嬖臣君集解徐廣曰一作吾。索隱諸衆讒嬖臣爲一句君字宜屬下文。德誠施皆清矣。禹曰然。帝即不時布同善惡則毋功。集解孔安國曰帝用臣不是則賢愚並位優劣共流故也。

帝曰正義此二字及下禹曰尚書並無。太史公有四字帝及禹相答極爲次序當應別見書。毋若丹朱傲維慢游是好毋水行舟朋淫于家集解鄭玄曰朋淫淫門內。用絕其世。予不能順是。禹曰予辛壬娶塗山辛壬癸甲生啟予不子集解孔安國曰塗山國名。辛日娶妻至于甲四日復往治水。索隱杜預云塗山在壽春東北皇甫謐云今九江當塗有禹廟則塗山在江南也。系本曰塗山氏女名女媧是禹娶塗山氏號女媧也。又按尚書云娶于塗山辛壬癸甲啟呱呱而泣予弗子。今此云辛壬娶塗山癸甲生啟蓋今文尚書脫漏太史公取以爲言亦不稽其本意。豈有辛壬娶妻經二日生子不經之甚。正義此五字爲一句。禹辛日娶至甲四日往理水及生啟不入門我不得名子以故能成水土之功。又一云過門不入不得有子愛之心。帝繫云禹娶塗山氏之子謂之女媧是生啟也。以故能成水土功。輔成五服至于五千里州十二師外薄四海集解孔安國曰薄迫。言至海也。正義爾雅云九夷八狄七戎六蠻謂之四海。釋名云海晦也。按夷蠻晦昧無知故云四海也。咸建五長集解孔安國曰諸侯五國立賢者一人爲方伯謂之五長以相統治。各道有功。苗頑不即功集解孔安國曰三苗頑凶不得就官善惡分別。帝其念哉。帝曰道吾德乃女功序之也。

皋陶於是敬禹之德令民皆則禹。不如言刑從之。索隱謂不用命之人則亦以刑罰而從之。舜德大明。

於是夔行樂正義若今太常卿也。祖考至羣后相讓鳥獸翔舞簫韶九成鳳皇來儀集解孔安國曰簫韶舜樂名。備樂九奏而致鳳皇也。百獸率舞百官信諧。帝用此作歌曰陟天之命維時維幾。集解孔安國曰奉正天命以臨民惟在順時惟在慎微。乃歌曰股肱喜哉元首起哉百工熙哉集解孔安國曰股肱之臣喜樂盡忠君之治功乃起百官之業乃廣。皋陶拜手稽首揚言曰念哉集解鄭玄曰使羣臣念帝之戒。率爲興事慎乃憲敬哉集解孔安國曰率臣下爲起治之事當慎汝法度敬其職。乃更爲歌曰元首明哉股肱良哉庶事康哉舜又歌曰元首叢脞哉股肱惰哉萬事墮哉集解孔安國曰叢脞細碎無大略也。君如此則臣懈惰萬事墮廢也。帝拜曰然往欽哉於是天下皆宗禹之明度數聲樂集解徐廣曰舜本紀云禹乃興九韶之樂。爲山川神主。

帝舜薦禹於天爲嗣。十七年集解劉熙曰若此則舜格于文祖三年之後攝禹使得祭祀與而帝舜崩。三年喪畢禹辭辟舜之子商均於陽城。集解劉熙曰今潁川陽城是也。天下諸侯皆去商均而朝禹。禹於是遂即天子位集解皇甫謐曰都平陽或在安邑或在晉陽。南面朝天下國號曰夏后姓姒氏。集解禮緯曰祖以吞薏苡生。

帝禹立而舉皋陶薦之且授政焉而皋陶卒。正義帝王紀云皋陶生於曲阜。曲阜偃地故帝因之而以賜姓曰偃。堯禪舜命之作士。舜禪禹禹即帝位以咎陶最賢薦之於天將有禪之意。未及禪會皋陶卒。括地志云咎繇墓在壽州安豐縣南一百三十里故六城東東都陂內大冢也。封皋陶之後於英六集解徐廣曰史記皆作英字而以英布是此苗裔。索隱地理志六安國六縣咎繇後偃姓所封國。英地闕不知所在以爲黥布是其後也。正義英蓋蓼也。括地志云光州固始縣本春秋時蓼國。偃姓皋陶之後也。左傳云子燮滅蓼。太康地志云蓼國先在南陽故縣今豫州郾縣界故胡城是後徙於此。括地志云故六城在壽州安豐縣南一百三十二里。春秋文五年秋楚成大心滅之。或在許。集解皇覽曰皋陶冢在廬江六縣。索隱許在潁川。正義括地志云許故城在許州許昌縣南三十里本漢許縣故許國也。而后舉益任之政。

十年帝禹東廵狩至于會稽而崩。集解皇甫謐曰年百歲也。以天下授益。三年之喪畢益讓帝禹之子啟而辟居箕山之陽。集解孟子陽字作陰。劉熙曰崈髙之北。正義按陰即陽城也。括地志云陽城縣在箕山北十三里。又恐箕字誤本是嵩字而字相似。其陽城縣在嵩山南二十三里則爲嵩山之陽也。禹子啟賢天下屬意焉。及禹崩雖授益益之佐禹日淺天下未洽。故諸侯皆去益而朝啟曰吾君帝禹之子也。於是啟遂即天子之位是爲夏后帝啟。

夏后帝啟禹之子其母塗山氏之女也。

有扈氏不服集解地理志曰扶風鄠縣是扈國。索隱地理志曰扶風縣鄠是扈國。正義括地志云雍州南鄠縣本夏之扈國也。地理志云鄠縣古扈國有戶亭。訓纂云戶扈鄠三字一也古今字不同耳。啟伐之大戰於甘。集解馬融曰甘有扈氏南郊地名。索隱夏啟所伐鄠南有甘亭。將戰作甘誓乃召六卿申之。集解孔安國曰天子六軍其將皆命卿也。啟曰嗟六事之人集解孔安國曰各有軍事故曰六事。予誓告女有扈氏威侮五行怠棄三正集解鄭玄曰五行四時盛德所行之政也。威侮暴逆之。三正天地人之正道。天用勦絕其命。集解孔安國曰勦截也。今予維共行天之罰。集解孔安國曰共奉也。左不攻于左右不攻于右女不共命。集解鄭玄曰左車左。右車右。御非其馬之政女不共命。集解孔安國曰御以正馬爲政也。三者有失皆不奉我命也。用命賞于祖集解孔安國曰天子親征必載遷廟之祖主行。有功即賞祖主前示不專也。不用命僇于社集解孔安國曰又載社主謂之社事。奔北則僇之社主前。社主陰陰主殺也。予則帑僇女。集解孔安國曰非但止身辱及女子言恥累之。遂滅有扈氏。天下咸朝。

夏后帝啟崩集解徐廣曰皇甫謐曰夏啟元年甲辰十年癸丑崩。子帝太康立。帝太康失國集解孔安國曰盤于遊田不恤民事爲羿所逐不得反國。昆弟五人索隱皇甫謐云號五觀也。須于洛汭作五子之歌。集解孔安國曰太康五弟與其母待太康于洛水之北怨其不反故作歌。

太康崩弟中康立是爲帝中康。帝中康時羲和湎淫廢時亂日。集解孔安國曰羲氏和氏掌天地四時之官。太康之後沈湎于酒廢天時亂甲乙也。胤往征之作胤征。集解孔安國曰胤國之君受王命往征之。鄭玄曰胤臣名。

中康崩子帝相立。帝相崩子帝少康立。索隱左傳魏莊子曰昔有夏之衰也后羿自鉏遷于窮石因夏人而代夏政。恃其射也不修人事而信用伯明氏之讒子寒浞。浞殺羿烹之以食其子子不忍食殺于窮門。浞因羿室生澆及豷。使澆滅斟灌氏及斟尋氏而相爲澆所滅后緡歸于有仍生少康。有夏之臣靡自有鬲收二國之燼以滅浞而立少康。少康滅澆于過后杼滅豷于戈有窮遂亡。然則帝相自被篡殺中閒經羿浞二氏蓋三數十年。而此紀總不言之直云帝相崩子少康立疏略之甚。正義帝王紀云帝羿有窮氏未聞其先何姓。帝嚳以上世掌射正。至嚳賜以彤弓素矢封之於鉏爲帝司射歷虞夏。羿學射於吉甫其臂長故以善射聞。及夏之衰自鉏遷于窮石因夏民以代夏政。帝相徙于商丘依同姓諸侯斟尋。羿恃其善射不修民事淫于田獸棄其良臣武羅伯姻熊髡尨圉而信寒浞。寒浞伯明氏之讒子伯明后以讒棄之而羿以爲己相。寒浞殺羿於桃梧而烹之以食其子。其子不忍食之死于窮門。浞遂代夏立爲帝。寒浞襲有窮之號因羿之室生奡及豷。奡多力能陸地行舟。使奡帥師滅斟灌斟尋殺夏帝相封奡於過封豷於戈。恃其詐力不恤民事。初奡之殺帝相也妃有仍氏女曰后緡歸有仍生少康。初夏之遺臣曰靡事羿羿死逃於有鬲氏收斟尋二國餘燼殺寒浞立少康滅奡於過后杼滅豷於戈有窮遂亡也。按帝相被篡歷羿浞二世四十年而此紀不說亦馬遷所爲疏略也。奡音五告反。豷音許器反。括地志云故鉏城在滑州韋城縣東十里。晉地記云河南有窮谷蓋本有窮氏所遷也。括地志云商丘今宋州也。斟灌故城在青州壽光縣東五十四里。斟尋故城今青州北海縣是也。故過鄉亭在萊州掖縣西北二十里本過國地。故鬲城在洛州密縣界。杜預云國名今平原鬲縣也。戈在宋鄭之閒也。寒國在北海平壽縣東寒亭也。伯明其君也。臣瓚云斟尋在河南蓋後遷北海也。汲冢古文云太康居斟尋羿亦居之桀又居之。尚書云太康失邦兄弟五人須于洛汭。此即太康居之爲近洛也。又吳起對魏武侯曰夏桀之居左河濟右太華伊闕在其南羊腸在其北。又周書度邑篇云武王問太公吾將因有夏之居即河南是也。括地志云故鄩城在洛州鞏縣西南五十八里蓋桀所居也。陽翟縣又是禹所封爲夏伯。帝少康崩子帝予索隱音佇。系本云季佇作甲者也。左傳曰杼滅豷于戈。國語云杼能帥禹者也。立。帝予崩子帝槐索隱音回。系本作帝芬。立。帝槐崩子帝芒索隱音亡。鄒誕生又音荒也。立。帝芒崩子帝泄立。帝泄崩子帝不降索隱系本作帝降。立。帝不降崩弟帝扃立。帝扃崩子帝厪索隱音覲。鄒誕生又音勤。立。帝厪崩立帝不降之子孔甲是爲帝孔甲。帝孔甲立好方鬼神事淫亂。夏后氏德衰諸侯畔之。天降龍二有雌雄孔甲不能食正義音寺。未得豢龍氏。集解賈逵曰豢養也。穀食曰豢。陶唐既衰其后有劉累集解服虔曰后劉累之爲諸侯者夏后賜之姓。正義括地志云劉累故城在洛州緱氏縣南五十五里乃劉累之故地也。學擾龍集解應劭曰擾音柔。擾馴也。能順養得其嗜慾。于豢龍氏以事孔甲。孔甲賜之姓曰御龍氏集解服虔曰御亦養。受豕韋之後。集解徐廣曰受一作更。駰案賈逵曰劉累之後至商不絕以代豕韋之後。祝融之後封於豕韋殷武丁滅之以劉累之後代之。索隱按系本豕韋防姓。龍一雌死以食夏后。夏后使求懼而遷去。集解賈逵曰夏后既饗而又使求致龍劉累不能得而懼也。傳曰遷於魯縣。

孔甲崩子帝皋立。帝皋崩集解左傳曰皋墓在殽南陵。子帝發立。帝發崩子帝履癸立是爲桀。索隱桀名也。按系本帝皋生發及桀。此以發生桀皇甫謐同也。帝桀之時集解謐法賊人多殺曰桀。自孔甲以來而諸侯多畔夏桀不務德而武傷百姓百姓弗堪。迺召湯而囚之夏臺索隱獄名。夏曰均臺。皇甫謐云地在陽翟是也。已而釋之。湯修德諸侯皆歸湯湯遂率兵以伐夏桀。桀走鳴條集解孔安國曰地在安邑之西。鄭玄曰南夷地名。遂放而死。集解徐廣曰從禹至桀十七君十四世。駰案汲冢紀年曰有王與無王用歲四百七十一年矣。索隱徐廣曰從禹至桀十七君十四世。案汲冢紀年曰有王與無王用歲四百七十一年。正義括地志云廬州巢縣有巢湖即尚書成湯伐桀放於南巢者也。淮南子云湯敗桀於歷山與末喜同舟浮江奔南巢之山而死。國語云滿於巢湖。又云夏桀伐有施施人以妺喜女焉。女音女慮反。桀謂人曰吾悔不遂殺湯於夏臺使至此。湯乃踐天子位代夏朝天下。湯封夏之後正義括地志云夏亭故城在汝州郟城縣東北五十四里蓋夏后所封也。至周封於杞也。正義括地志云汴州雍丘縣古杞國城也。周武王封禹後號東樓公也。

太史公曰禹爲姒姓其後分封用國爲姓故有夏后氏有扈氏有男氏斟尋氏集解徐廣曰一作斟氏尋氏。彤城氏襃氏費氏索隱系本男作南尋作鄩費作弗而不云彤城及襃。按周有彤伯蓋彤城氏之後。張敖地理記云濟南平壽縣其地即古斟尋國。又下云斟戈氏按左傳系本皆云斟灌氏。杞氏繒氏辛氏冥氏斟氏戈氏。孔子正夏時學者多傳夏小正云。集解禮運稱孔子曰我欲觀夏道是故之杞而不足徵也吾得夏時焉。鄭玄曰得夏四時之書其存者有小正。索隱小正大戴記篇名。正征二音。自虞夏時貢賦備矣。或言禹會諸侯江南計功而崩因葬焉命曰會稽。會稽者會計也。集解皇覽曰禹冢在山陰縣會稽山上。會稽山本名苗山在縣南去縣七里。越傳曰禹到大越上苗山大會計爵有德封有功因而更名苗山曰會稽。因病死葬葦棺穿壙深七尺上無瀉泄下無邸水檀髙三尺土階三等周方一畝。呂氏春秋曰禹葬會稽不煩人徒。墨子曰禹葬會稽衣裘三領桐棺三寸。地理志云山上有禹井禹祠相傳以爲下有羣鳥耘田者也。索隱抵至也音丁禮反。葦棺者以葦爲棺。謂蘧蒢而斂非也。禹雖儉約豈萬乘之主而臣子乃以蘧蒢裹尸乎墨子言桐棺三寸差近人情。正義括地志云禹陵在越州會稽縣南十三里。廟在縣東南十一里。

索隱述贊堯遭鴻水黎人阻飢。禹勤溝洫手足胼胝。言乘四載動履四時。娶妻有日過門不私。九土既理玄圭錫茲。帝啟嗣立有扈違命。五子作歌太康失政。羿浞斯侮夏室不競。降于孔甲擾龍乖性。嗟彼鳴條其終不令
