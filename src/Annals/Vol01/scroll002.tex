帝顓頊高陽者|[%
皇甫謐曰都帝丘今東郡濮陽是也。%
索隱曰宋衷云顓頊名高陽有天下號%
%
也張晏曰高陽%
者所興地名也%
]黃帝之孫而昌意之子也靜淵以%
%
有謀疏通而知事養材以任地|[%
索隱曰言能養材%
物以任地大戴禮%
%
作養%
財%
]載時以象天|[%
索隱曰載行也言行四時以象天大%
戴禮作履時以象天履亦踐而行也%
%
]依鬼神以制義|[%
索隱曰鬼神聦明正直當盡心敬事因%
制尊卑之義故禮曰降于祖廟之謂仁%
%
義是也。正義曰鬼之靈者曰神也鬼神謂山川之神也%
能興雲致雨潤養萬物也故己依馮之剬義也剬古制字%
]治%
%
氣以教化|[%
索隱曰謂理四時五行%
之氣以教化萬人也%
]絜誠以祭祀北至%
%
于幽陵|[%
正義曰%
幽州也%
]南至于交趾|[%
正義曰趾音%
止交州也%
]西至于流%
%
沙|[%
地理志曰流沙在張掖居延縣。正義曰濟渡也括%
地志云居延海南甘州張掖縣東北千六十四里是%
]東至%
%
于蟠木|[%
海外經曰東海中有山焉名曰度索上有大桃樹%
屈蟠三千里東北有門名曰鬼門萬鬼所聚也天%
%
帝使神人守之一名鬱壘主閱領萬鬼若害%
人之鬼以葦索縛之射以桃弧投虎食也%
]動靜之物|[%
正%
義%
%
曰動物謂鳥獸之類%
靜物謂草木之類%
]大小之神|[%
正義曰大謂五嶽四%
瀆小謂丘陵墳衍%
]日%
%
月所照莫不砥屬|[%
王肅曰砥平也四遠皆平而來服屬%
。索隱曰依王肅音止屬據大戴禮%
%
作砥%
礪也%
]帝顓頊生子曰窮蟬|[%
索隱曰系本作窮係宋衷%
云一云窮係謚也。正義%
%
曰帝舜之%
高祖也%
]顓頊崩|[%
皇甫謐曰在位七十八年年九十八皇%
覽曰顓頊冢在東郡濮陽頓丘城門外%
%
廣陽里中頓丘者城門名頓丘道。索隱曰皇甫謐云據左%
氏歳在鶉火而崩葬東郡又山海經曰顓頊葬鮒魚山之陽%
%
九嬪葬%
其隂也%
]而玄囂之孫高辛立是爲帝嚳%
