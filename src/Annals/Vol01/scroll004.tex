帝堯者|[%
謚法曰翼善傳聖曰堯。索隱曰堯謚也放勳名%
帝嚳之子姓伊祁氏案皇甫謐云堯初生時其母%
%
在三阿之南寄於伊長孺之家故從母所居爲姓也。正義%
曰徐廣云號陶唐帝王紀云堯都平陽於詩爲唐國徐才宗%
%
國都城記云唐國帝堯之裔子所封其北帝夏禹都漢曰太%
原郡在古冀州太行恆山之西其南有晉水括地志云今晉%
%
州所理平陽故城是也%
平陽河水一名晉水也%
]放勳|[%
徐廣曰號陶唐皇甫謐曰堯%
以甲申歳生甲辰即帝位甲%
%
午徵舜甲寅舜代行天子事辛%
巳崩年百十八在位九十八年%
]其仁如天|[%
索隱曰如天%
之涵養也%
]其%
%
知如神|[%
索隱曰如神%
之微妙也%
]就之如日|[%
索隱曰如日之照臨人%
咸依就之若葵藿傾心%
%
以向%
日也%
]望之如雲|[%
索隱曰如雲之覆渥言德化廣大而浸潤%
生人人咸仰望之故曰如百穀之仰膏雨%
%
也%
]富而不驕貴而不舒|[%
索隱曰舒猶慢也%
大戴禮作不豫%
]黃收純%
%
衣|[%
徐廣曰純一作絞駰案太古冠冕圖云夏名冕曰收禮記%
曰野夫黃冠鄭玄曰純衣士之祭服。索隱曰收冕名其%
色黃故曰黃收象古%
質素也純讀曰緇%
]彤車乘白馬能明馴德|[%
徐廣曰%
馴古訓%
%
字。索隱曰史記馴字徐廣皆讀曰訓訓順也言聖德能順%
人也案尚書作俊德孔安國云能明用俊德之士與此文意%
%
別%
也%
]以親九族九族旣睦便章百姓|[%
徐廣曰下云便程%
東作然則訓平爲%
%
便也駰案尚書並作平字孔安國曰百姓百官鄭玄曰百姓%
羣臣之父子兄弟。索隱曰古文尚書作平此文盖讀平爲%
%
浦耕反平旣訓便因作便章其今文作辯章古平字亦%
作便音婢緑反便則訓辯遂爲辯章鄒誕生本亦同也%
]百姓%
%
昭明合和萬國乃命羲和|[%
孔安國曰重黎之後羲氏%
和氏世掌天地之官。正%
%
義曰呂刑傳云重即羲黎即和雖別爲氏族而出自重黎也%
按聖人不獨治必須賢輔乃命相天地之官若周禮天官卿%
%
地官%
卿也%
]敬順昊天|[%
正義曰敬猶恭勤也元氣昊然廣大故云%
昊天釋天云春爲蒼天夏爲昊天秋爲旻%
%
天冬爲上天而獨言昊天者以%
堯能敬天大故以昊大言之%
]數法日月星辰|[%
正義曰%
曆數之%
%
法日之甲乙月之大小昏明遞中之星日%
月所會之辰定其天數以爲一歳之曆%
]敬授民時|[%
索隱%
曰尚%
%
書作曆象日月則此言數法是訓曆象二字謂命羲和以曆%
數之法觀察日月星辰之早晚以敬授人時也。正義曰尚%
%
書考靈耀云主春者張昏中可以種稷主夏者火昏中可以%
種黍菽主秋者虛昏中可以種麥主冬者昴昏中可以收斂%
%
也天子視四星之中知民%
緩急故云敬授民時也%
]分命羲仲居郁夷曰暘谷|[%
%
尚書作嵎夷孔安國曰東表之地稱嵎夷日出於暘谷羲仲%
治東方之官。索隱曰史記舊本作湯谷今並依尚書字案%
%
淮南子曰日出湯谷浴於咸池則湯谷亦有他證明矣又下%
曰昧谷徐廣云一作柳柳亦日入處地名太史公博採經記%
%
而爲此史廣記異聞不必皆依尚書盖郁夷亦地之別名也%
。正義曰郁音隅陽或作暘禹貢青州云隅夷旣略按隅夷%
%
青州也堯命羲仲理東方青州隅夷之地日所出%
處名曰陽明之谷羲仲主東方之官若周禮春官%
]敬道日%
%
出便程東作|[%
孔安國曰敬道出日平均次序東作之事以%
務農也。索隱曰劉伯莊傳皆依古史作平%
%
秩音然尚書大傳曰辯秩東作則是訓秩爲程言便課其作%
程者也。正義曰道音導便程並如字後同導訓也三春主%
%
東故言日出耕作在春故言東作命羲%
仲恭勤道訓萬民東作之事使有程期%
]日中星鳥以殷%
%
中春|[%
孔安國曰日中謂春分之日也鳥南方朱鳥七宿也%
殷正也春分之昏鳥星畢見以正仲春之氣節轉以%
%
推孟季則可知也。正義%
曰下中音仲夏秋冬並同%
]其民析鳥獸字㣲|[%
孔安國曰%
春事旣起%
%
丁壯就功言其民老壯分析也乳化曰%
字尚書㣲作尾字說文云尾交接也%
]申命羲叔居南%
%
交|[%
孔安國曰夏與春交此治南方之官也。索隱曰孔註未%
是然則冬與秋交何故下無其文且東嵎夷西昧谷北幽%
%
都三方皆言地而夏獨不言地乃云與春交斯不例之甚也%
然南方地有名交阯者或古文略舉一字名地南交則是交%
%
阯不疑也。正義曰羲叔主%
南方官若周禮夏官卿也%
]便程南譌敬致|[%
孔安國曰%
譌化也平%
%
序分南方化育之事敬行其教以致其功也。索隱曰爲依%
字讀春言東作夏言南爲皆是耕作營爲勸農之事孔安國%
%
強讀爲訛字雖則訓化解釋亦甚紆回也。正義曰爲%
音于僞反命羲叔宜恭勤民事致其種殖使有程期也%
]日永%
%
星火以正中夏|[%
孔安國曰永長也謂夏至之日火蒼龍%
之中星舉中則七星見可知也以正中%
%
夏之節馬融王肅謂日長晝%
漏六十刻鄭玄曰五十五刻%
]其民因鳥獸希革|[%
孔安國%
曰因謂%
%
老弱因就在田之丁壯以助農也夏%
時鳥獸毛羽希少改易也革改也%
]申命和仲居西土|[%
%
徐廣曰一無土字以爲西者今天水之西縣也%
駰案鄭玄曰西者隴西之西今人謂之兌山%
]曰昧谷|[%
徐%
廣%
%
曰一作柳谷駰案孔安國曰日入于谷而天下冥故曰昧谷%
此居治西方之官掌秋天之政也。正義曰和仲主西方之%
%
官若周禮%
秋官卿也%
]敬道日入便程西成|[%
孔安國曰秋西%
方萬物成也%
]夜中%
%
星虛以正中秋|[%
孔安國曰春言日秋言夜互相備也虛%
玄武之中星亦言七星皆以秋分日見%
%
以正三秋也。索隱曰虛舊依字讀而鄒%
誕生音墟案虛星主墳墓鄒氏或得其理%
]其民夷易鳥%
%
獸毛毨|[%
孔安國曰夷平也老壯者在田與%
夏平也毨理也毛更生曰毨理%
]申命和叔居%
%
北方曰幽都|[%
孔安國曰北稱幽都謂所聚也。索隱曰案%
山海經曰北海之內有山名幽都盖是也。%
%
正義曰按北方幽州隂聚之也命和%
叔居理之北方之官若周禮冬官卿%
]便在伏物|[%
索隱曰使%
和叔察北%
%
方藏伏之物謂人畜積聚等冬皆藏伏尸子亦曰北方者伏%
方也尚書作平在朔易今案大傳云便在伏物太史公據之%
%
而%
書%
]日短星昴以正中冬|[%
孔安國曰日短冬至之日也昴%
白虎之中星亦以七星並見以%
%
正冬莭也馬融王肅謂日短晝%
漏四十刻鄭玄曰四十五刻非%
]其民燠鳥獸氄毛|[%
徐%
廣%
%
曰氄音茸駰案孔安國曰民入室%
處鳥獸皆生氄毳細毛以自溫也%
]歳三百六十六日以%
%
閏月正四時|[%
索隱曰夫周天三百六十五度四分度之一%
是天度數也而日行遲一歳一周天月行疾%
%
一月一周天日一日行一度月一日行十三度十九分度之%
七至二十九日半彊月行天一匝又逐及日而與會一年十%
%
二會是爲十二月每月二十九日過半年分出小月六是每%
歳餘六日又大歳三百六十六日小歳三百六十五日舉全%
%
數云六十六日其實一歳唯餘十一日弱未滿三歳已成一%
月則置閏若三年不置閏則正月爲二月九年差三月則以%
%
春爲夏十七年差六月則四時皆反以此四時不%
正歳不成矣故傳曰歸餘於終事則不悖是也%
]信飭|[%
徐%
廣%
%
曰古%
勑字%
]百官衆功皆興堯曰誰可順此事|[%
正義曰言%
將登用之%
%
嗣位%
也%
]放齊曰嗣子丹朱開明|[%
孔安國曰放齊臣名。%
正義曰放音方往反鄭%
%
玄云帝堯胤嗣之子名曰丹朱開明也按開解而達也帝王%
紀云堯娶散宜氏女曰女皇生丹朱汲冢紀年云后稷放帝%
%
子丹朱范汪荆州記云丹水縣在丹川堯子朱之所封也括%
地志云丹水故城在鄧州內鄉縣西南百三十里丹水故爲%
%
縣%
]堯曰吁頑凶不用|[%
孔安國曰吁疑怪之辭。正義曰%
左傳云口不道忠信之言爲嚚心%
%
不則德義之經爲頑凶訟也言丹%
朱心旣頑嚚又好争訟不可用之%
]堯又曰誰可者讙兠%
%
曰共工旁聚布功可用|[%
孔安國曰讙兠臣名鄭玄曰共%
工水官名。正義曰兠音斗侯%
%
反%
]堯曰共工善言其用僻似恭漫天不可|[%
正義曰%
漫音莫%
%
干反共工善爲言語用意邪僻也%
似於恭敬罪惡漫天不可用也%
]堯又曰嗟四嶽|[%
鄭玄%
曰四%
%
嶽四時官主方嶽之事。正義曰嗟嘆鴻水問四嶽誰能理%
也孔安國云四嶽即上羲和四子也分掌四嶽之諸侯故稱%
%
焉%
]湯湯洪水滔天浩浩懷山襄陵|[%
孔安國曰懷包%
襄上也。正義%
%
曰湯音商今讀如字蕩蕩廣平之貌言水奔突有所滌除地%
上之物爲水漂流蕩蕩然按懷藏包裹之義故懷爲包釋言%
%
以襄爲駕駕乘牛馬皆在上也言%
水襄上乘陵浩浩盛大勢若漫天%
]下民其憂有能使%
%
治者皆曰鯀可|[%
馬融曰鯀%
臣名禹父%
]堯曰鯀負命毀族不%
%
可|[%
正義曰負音佩依字通負違也族類也鯀性很戾%
違負教命毀敗善類不可用也詩云貪人敗類也%
]嶽曰%
%
异哉試不可用而已|[%
正義曰异音異孔安國云异已也%
退也言餘人盡已唯鯀可試無成%
%
乃%
退%
]堯於是聽嶽用鯀九歳功用不成|[%
正義曰爾雅%
釋天云載歳%
%
也夏曰祀周曰年唐虞曰載李廵云各自紀事不相襲也孫%
炎云歳取星行一次也祀取四時祭祀一訖也年取禾穀一%
%
熟也載取萬物始更終也載者年之別名故以載爲年也按%
功用不成水害不息故放退也至明年得舜乃殛之羽山而%
%
用其子%
禹也%
]堯曰嗟四嶽朕在位七十載汝能庸命踐%
%
朕位|[%
鄭玄曰言汝諸侯之中有能順事用天命者入處我位%
統治天子之事者乎。正義曰孔安國云堯年十六以%
%
唐侯升爲天子在位七十%
載時八十六老將求代也%
]嶽應曰鄙德忝帝位|[%
正義%
曰四%
%
嶽皆云鄙俚無德若便行天子%
事是辱帝位言己等不堪也%
]堯曰悉舉貴戚及疏遠%
%
隱匿者衆皆於堯曰有矜在民間曰虞舜|[%
孔安%
國曰%
%
無妻曰矜。正%
義曰矜古頑反%
]堯曰然朕聞之其何如嶽曰盲者%
%
子父頑母嚚弟傲能和以孝烝烝治不至姦|[%
正%
義%
%
曰烝之升反進也言父頑母嚚弟傲舜%
皆和以孝進之於善不至於姦惡也%
]堯曰吾其試哉|[%
%
正義曰欲以二女試%
舜觀其理家之道也%
]於是堯妻之二女|[%
正義曰妻音七%
計反二女娥皇%
%
女英也娥皇無子女英生商均%
舜升天子娥皇爲后女英爲妃%
]觀其德於二女|[%
正義曰%
視其爲%
%
德行於二女以%
理家而觀國也%
]舜飭下二女於嬀汭|[%
孔安國曰舜所居%
嬀水之汭。索隱曰%
%
列女傳云二女長曰娥皇次曰女英系本作女瑩大戴禮作%
女匽皇甫謐云嬀水在河東虞縣歷山西汭水涯也猶洛汭%
%
渭汭然也。正義曰飭音勑下音胡亞反汭音芮舜能整齊%
二女以義理下二女之心於嬀汭使行婦道於虞氏也括地%
%
志云嬀源汭水出蒲州河東南山許慎云水涯曰汭按地記%
云河東郡青山東山中有二泉下南流者嬀水北流者汭水%
%
二水異源合流出谷西注河嬀水北曰汭也又云河東縣二%
里故蒲坂城舜所都也城中有舜廟城外有舜宅入二妃壇%
%
]如婦禮堯善之乃使舜慎和五典|[%
鄭玄曰五典五%
教也盖試以司%
%
徒之%
職%
]五典能從乃徧入百官百官時序賔於四門%
%
四門穆穆諸侯遠方賔客皆敬|[%
馬融曰四門四方%
之門諸侯羣臣朝%
%
者舜賔迎之%
皆有美德也%
]堯使舜入山林川澤暴風雷雨舜行%
%
不迷|[%
索隱曰尚書云納于大麓穀梁傳云林屬於山曰麓%
是山足曰麓故此以爲入山林不迷孔氏以麓訓録%
%
言令舜大録萬機%
之政與此不同%
]堯以爲聖召舜曰女謀事至而%
%
言可績三年矣|[%
鄭玄曰三年者賔%
四門之後三年也%
]女登帝位舜讓%
%
於德不懌|[%
徐廣曰音亦今文尚書作不怡怡懌也。索隱%
曰古文作不嗣今文作不怡怡即懌也謂辭讓%
%
於德不堪所以心意不悅懌%
也俗本作澤誤爾亦當爲懌%
]正月上日|[%
馬融曰上日朔日%
也。正義曰鄭玄%
%
云帝王易代莫不改正建朔堯正建%
子此時未改故依堯正月上日也%
]舜受終於文祖文%
%
祖者堯大祖也|[%
鄭玄曰文祖者五府之大名猶周之明%
堂。索隱曰尚書帝命驗曰五府五帝%
%
之廟蒼曰靈府赤曰文祖黃曰祖計白曰顯紀黑曰玄矩唐%
虞謂之五府夏謂世室殷謂重屋周謂明堂皆祀五帝之所%
%
也。正義曰舜受堯終帝之事於文祖也尚書帝命驗云帝%
者承天立府以尊天重象也五府者黃曰神斗注云唐虞謂%
%
之天府夏謂之正室殷謂之重室周謂之明堂皆祀五帝之%
所也文祖者赤帝熛怒之府名曰文祖火精光明文章之祖%
%
故謂之文祖周曰明堂神斗者黃帝含樞紐之府名曰神斗%
斗主也土精澄靜四行之主故謂之神斗周曰太室顯紀者%
%
白帝招拒之府名顯紀紀法也金精斷割萬物故謂之顯紀%
周曰緫章玄矩者黑帝光紀之府名曰玄矩矩法也水精玄%
%
味能權輕重故謂之玄矩周曰玄堂靈府%
者蒼帝靈威仰之府名曰靈府周曰青陽%
]於是帝堯老%
%
命舜攝行天子之政以觀天命舜乃在璿璣玉%
%
衡以齊七政|[%
鄭玄曰璿璣玉衡渾天儀也七政日月五星%
也。正義曰說文云璿赤玉也按舜雖受堯%
%
命猶不自安更以璿璣玉衡以正天文璣爲運轉衡爲橫簫%
運璣使動於下以衡望之是王者正天文器也觀其齊與不%
%
齊今七政齊則己受禪爲是蔡邕云玉衡長八尺孔徑一寸%
下端望之以視星宿並縣璣以象天而以衡望之轉璣窺衡%
%
以知星宿璣徑八尺圓周二尺五寸而強也鄭玄云運轉者%
爲璣持正者爲衡尚書大傳云政者齊中也謂春秋冬夏天%
%
文地理人道所以爲政也道正%
而萬事順成故天道政之大也%
]遂類于上帝|[%
鄭玄曰禮%
祭上帝於%
%
圜丘。正義曰五經異義云非時祭天謂之類言以事類告%
也時舜告攝非常祭也王制云天子將出類于上帝鄭玄云%
%
昊天上帝謂天皇%
大帝北辰之星%
]禋于六宗|[%
鄭玄曰六宗星辰司中司%
命風師雨師也駰案六宗%
%
義衆矣愚謂鄭說爲長。正義曰周語云精意以享曰禋也%
孫炎云禋絜敬之祭也按星五星緯也辰日月所會十二次%
%
也司中司命文昌第五第四星也風師箕星也雨師畢星也%
孔安國云四時寒暑也日月星也水旱也禮祭法云埋少牢%
%
於大昭祭時也禳祈於坎壇祭寒暑也王宮祭日也夜明祭%
月也幽禜祭星雩禜祭水旱也司馬彪續漢書云安帝立六%
%
宗祀於洛陽城西北亥地禮比大社魏因之至晉初%
荀顗言新祀以六宗之神諸家說不同乃廢之也%
]望于%
%
山川|[%
正義曰望者遙望而祭山川也山川%
五嶽四瀆也爾雅云梁山晉望也%
]辯於羣神|[%
徐%
廣%
%
曰辯音班駰案鄭玄曰羣神若丘陵%
墳衍。正義曰辯音遍謂祭羣神也%
]揖五瑞擇吉月日%
%
見四嶽諸牧班瑞|[%
馬融曰揖斂也五瑞公侯伯子男所%
執以爲瑞信也堯將禪舜使羣牧斂%
%
之使舜親往班之。正義曰揖音集周禮典瑞云王執鎮圭%
尺二寸公執桓圭九寸侯執信圭七寸伯執躬圭五寸子執%
%
穀璧男執蒲璧皆五寸言五瑞者王不在中也孔文祥云宋%
末會稽修禹廟於廟庭山土中得五等圭璧百餘枚形與周%
%
禮同皆短小此即禹會諸侯於會稽執%
以禮山神而埋之其璧今猶有在也%
]歳二月東廵狩%
%
至於岱宗柴|[%
馬融曰舜受終後五年之二月鄭玄曰建卯%
之月也柴祭東嶽者考績柴燎也。正義曰%
%
按旣班瑞羣后即東廵者守土之諸侯會岱宗之嶽焚柴告%
至也王者廵狩以諸侯自專一國威福任己恐其壅遏上命%
%
澤不下流故廵行問人疾苦也風俗通云太山之尊者一曰%
岱宗始也長也萬物之始隂陽交代故爲五岳之長也按二%
%
月仲月也仲中%
也言得其中也%
]望秩於山川|[%
正義曰乃以秩望祭東方%
諸侯境內之名山大川也%
%
言秩者五岳視三%
公四瀆視諸侯%
]遂見東方君長合時月正日|[%
鄭玄%
曰協%
%
正四時之月數及日名備有失誤。正義曰旣見東方君長%
乃合同四時氣節月之大小日之甲乙使齊一也周禮太史%
%
掌正歳年以序事頒正朔於邦國則節氣晦朔皆天%
子頒之猶恐諸侯國異或不齊同因廵狩合正之%
]同律%
%
度量衡|[%
鄭玄曰用隂律度丈尺量斗斛衡斤兩也。正義%
曰律之十二律度之丈尺量之斗斛衡之斤兩皆%
%
使天下相同無制度長短輕重異也漢律志云虞書云同律%
度量衡所以齊遠近立民信也律有十二陽六爲律隂六爲%
%
呂律以統氣類物一曰黃鍾二曰太蔟三曰姑洗四曰蕤賔%
五曰夷則六曰無射呂以旅陽宣氣一曰林鍾二曰南呂三%
%
曰應鍾四曰大呂五曰夾鍾六曰中呂度者分寸尺丈引也%
所以度長短也本起黃鍾之管長以子穀秬黍中者一黍爲%
%
一分十分爲一寸十寸爲尺十尺爲丈十丈爲引而五度審%
矣量者龠合升斗斛也所以量多少也本起黃鍾之龠以子%
%
穀秬黍中者千有二百實爲一龠合龠爲合十合爲升十升%
爲斗十斗爲斛而五量嘉矣衡權者銖兩斤鈞石也所以秤%
%
物輕重也本起於黃鍾之一龠容千二百黍重十二銖二十%
四銖爲兩十六兩爲斤三十斤爲鈞四鈞爲石而五權謹矣%
%
衡平也%
權重也%
]脩五禮|[%
馬融曰吉凶賔軍嘉也。正義曰周禮以%
吉禮事邦國之鬼神祗以凶禮哀邦國之%
%
憂以賔禮親邦國以軍禮同邦國以嘉禮親萬民也尚書堯%
典云類于上帝吉禮也如喪考妣凶禮也羣后四朝賔禮也%
%
大禹謨云汝徂征軍禮也堯典%
云女于時嘉禮也女音女慮反%
]五玉|[%
鄭玄曰即五瑞也執%
之曰瑞陳列曰玉%
%
]三帛|[%
馬融曰三孤所執也鄭玄曰帛所以薦玉也必三者%
高陽氏後用赤繒高辛氏後用黑繒其餘諸侯皆用%
%
白繒。正義曰孔安國云諸侯世子執纁公之孤執玄附庸%
之君執黃也按三統紀推伏羲爲天統色尚赤神農爲地統%
%
色尚黑黃帝爲人統色尚白少昊黃帝子亦尚%
白故高陽氏又天統亦尚赤堯爲人統故用白%
]二生|[%
正義%
曰羔%
%
鴈也鄭玄注周禮大宗伯云羔小羊也取其羣不失其類也%
鴈取其候時而行也卿執羔大夫執鴈按羔鴈性馴可生爲%
%
贄%
]一死|[%
正義曰雉也馬融云一死雉士所執也按不%
可生爲贄故死雉取其守介死不失節也%
]爲摯|[%
%
馬融曰摯二生羔鴈卿大夫所執一死雉士所執。正義曰%
摯音至贄執也鄭玄云贄之言至所以自致也韋昭云贄六%
%
贄皮帛卿執羔大夫執鴈士%
執雉庶人執鹿工商執雞也%
]如五器卒乃復|[%
馬融曰五%
器上五玉%
%
五玉禮終則還之三帛已下不還%
也。正義卒音子律反復音伏%
]五月南廵狩八月西%
%
廵狩十一月北廵狩皆如初歸至于祖禰廟|[%
正%
義%
%
曰禰音乃禮反何休云%
生曰父死曰考廟曰禰%
]用特牛禮五歳一廵狩羣后%
%
四朝|[%
鄭玄曰廵狩之年諸侯見於方嶽之下%
其間四年四方諸侯分來朝於京師也%
]徧告以言|[%
%
正義曰徧音遍言遍%
告天子治理之言也%
]明試以功車服以庸|[%
正義曰孔安%
國云功成則%
%
錫車服以表%
顯其能用也%
]肇十有二州決川|[%
馬融曰禹平水土置九%
州舜以兾州之北廣大%
%
分置并州燕齊遼遠分燕置幽州分齊爲營州於%
是爲十二州也鄭玄曰更爲之定界濬水害也%
]象以典%
%
刑|[%
馬融曰言咎繇制五常之刑無犯之者但有其象無其人%
也。正義曰孔安國云象法也法用常刑用不越法也%
%
]流宥五刑|[%
馬融曰流放宥寛也一曰幼少二曰老耄三曰%
惷愚五刑墨劓剕宮大辟。正義曰孔安國云%
%
以流放之法寛五刑也鄭玄云三宥%
一曰弗識二曰過失三曰遺忘也%
]鞭作官刑|[%
馬融曰爲%
辨治官事%
%
者爲%
刑%
]扑作教刑|[%
鄭玄曰扑檟楚也%
扑爲教官爲刑者%
]金作贖刑|[%
馬融曰金%
黃金也意%
%
善功惡使出金贖%
罪坐不戒慎者%
]眚烖過赦|[%
鄭玄曰眚烖爲人作患害%
者也過失雖有害則赦之%
%
]怙終|[%
徐廣曰%
一作衆%
]賊刑|[%
鄭玄曰怙其姦邪終身%
以爲殘賊則用刑之%
]欽哉欽哉%
%
惟刑之靜哉|[%
徐廣曰今文云惟刑之謐哉爾雅曰謐靜也%
。索隱曰案古文作恤哉且今文是伏生口%
%
誦卹謐聲近%
遂作謐也%
]讙兠進言共工|[%
正義曰讙兠渾沌也共工%
窮竒也鯀檮杌也三苗饕%
%
餮也左傳云舜臣堯流四%
凶投諸四裔以禦魑魅也%
]堯曰不可而試之工師|[%
正%
義%
%
曰工師若今%
大匠卿也%
]共工果淫辟|[%
正義疋%
亦反%
]四嶽舉鯀治鴻%
%
水堯以爲不可嶽彊請試之試之而無功故百%
%
姓不便三苗|[%
馬融曰國名也。正義曰左傳云自古諸侯%
不用王命虞有三苗夏有觀扈孔安國云縉%
%
雲氏之後爲諸侯號饕餮也吳起云三苗之國左洞庭而右%
彭蠡按洞庭湖名在岳州巴陵西南一里南與青草湖連彭%
%
蠡湖名在江州潯陽縣東南五十二里以天子在北故洞庭%
在西爲左彭蠡在東爲右今江州鄂州岳州三苗之地也%
%
]在江淮荆州|[%
正義曰淮讀曰匯音胡罪反今彭蠡湖也本%
屬荆州尚書云南入于江東匯澤爲彭蠡是%
%
也%
]數爲亂於是舜歸而言於帝請流共工于幽%
%
陵|[%
馬融曰北裔也。正義曰尚書及大戴禮皆作幽州括地%
志云故龔城在檀州燕樂縣界故老傳云舜流共工幽州%
%
居此城神異經云西北荒有人焉人靣朱髴%
蛇身人手足而食五穀禽獸頑愚名曰共工%
]以變北狄|[%
徐%
廣%
%
曰變一作爕。索隱曰變謂變其形及衣服同於夷狄也徐%
廣云作爕爕和也。正義曰言四凶流四裔各於四夷放共%
%
工等爲中國%
之風俗也%
]放驩兠於崇山|[%
馬融曰南裔也。正義曰%
神異經云南方荒中有人%
%
焉人面鳥喙而有翼而手足扶翼而行食海中魚%
爲人很惡不畏風雨獸犯死乃休名曰讙兠也%
]以變南%
%
蠻遷三苗於三危|[%
馬融曰西裔也。正義曰括地志云%
三危山有峯故曰三危俗亦名卑羽%
%
山在沙州敦煌縣東南三十里神異經云西荒中有人焉面%
目手足皆人形而胳下有翼不能飛爲人饕餮淫逸無理名%
%
曰苗民又山海經云大荒北經黑%
水之北有人有翼名曰苗民也%
]以變西戎殛鯀於羽%
%
山|[%
馬融曰殛誅也羽山東裔也。正義曰殛音紀力反孔安%
國云殛竄放流皆誅也括地志云羽山在沂州臨沂縣界%
%
神異經云東方有人焉人形而身多毛自觧水%
土知通塞爲人自用欲爲欲息皆曰云是鯀也%
]以變東夷%
%
四辠而天下咸服堯立七十年得舜二十年而%
%
老令舜攝行天子之政薦之於天堯辟位凢二%
%
十八年而崩|[%
徐廣曰堯在位凢九十八年駰案皇覽曰堯%
冢在濟隂城陽劉向曰堯葬濟隂丘壠山呂%
%
氏春秋曰堯葬穀林皇甫謐曰穀林即城陽堯都平陽於詩%
為唐國。正義曰皇甫謐云堯即位九十八年通舜攝二十%
%
八年也凡年百一十七歳孔安國云堯壽百一十六歳括地%
志云堯陵在濮州雷澤縣西三里郭生述征記云城陽縣東%
%
有堯冢亦曰堯陵有碑是也括%
地志云雷澤縣本漢郕陽縣也%
]百姓悲哀如喪父母三%
%
年四方莫舉樂|[%
正義曰尚書三載四%
海遏密八音是也%
]以思堯堯知子%
%
丹朱之不肖|[%
索隱曰鄭玄曰肖似也不似言不如人也皇%
甫謐云堯娶散宜氏之女曰女皇生丹朱又%
%
有庶子九人%
皆不肖也%
]不足授天下於是乃權授舜|[%
索隱曰父%
子繼立常%
%
道也求賢而禪權道也權者反常而合道。%
正義曰五帝官天下老則禪賢故權試舜也%
]授舜則天下%
%
得其利而丹朱病授丹朱則天下病而丹朱得%
%
其利堯曰終不以天下之病而利一人而卒授%
%
舜以天下堯崩三年之喪畢舜讓辟丹朱於南%
%
河之南|[%
劉熙曰南河九河之最在南者。正義曰括地志%
云故堯城在濮州鄄城縣東北十五里竹書云昔%
%
堯德衰爲舜所囚也又有偃朱故城在縣西北十五里竹書%
云舜囚堯復偃塞丹朱使不與父相見也按濮州北臨漯大%
%
川也河在堯都之南故曰南河禹貢至于南河是也%
其偃朱城所居即舜讓避丹朱於河南之南處也%
]諸侯%
%
朝覲者不之丹朱而之舜獄訟者不之丹朱而%
%
之舜謳歌者不謳歌丹朱而謳歌舜舜曰天也%
%
夫而後之中國踐天子位焉|[%
劉熙曰天子之位不可%
曠年於是遂反格于文%
%
祖而當帝位帝王所%
都爲中故曰中國%
]是爲帝舜%
